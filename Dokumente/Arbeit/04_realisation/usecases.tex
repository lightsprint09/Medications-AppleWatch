\begin{usecase}
\addtitle{Usecase 1}{Der Patient wird an ein Medikament erinnert}
\addfield{Primärer Akteur}{Patient}
\addfield{Beschreibung}{Ein Erinnerungs Pop-Up erscheint auf dem Display und es wird ein Signal/eine Vibration ausgelost. Das Pop-Up zeigt ein Abbild des Medikaments, dessen Namen und die Uhrzeit, zu der es eingenommen werden soll.}
\addfield{Vorbedingung}{Es wurde ein Medikationsplan aus der DB auf die Uhr geladen.}
\additemizedfield{Ablauf}{
\item Das Pop-Up mit der Erinnerung erscheint auf dem Smartwatch-Display. Die Erinnerung beinhaltet Informationen zur Uhrzeit, Menge und Art der Medikation
\item Der Patient bestätigt, dass er das Medikament genommen hat
\item Gerät bestätigt visuell dass der Patient das Medikament als genommen markiert hat
}
\additemizedfield{Alternativablauf}{
\item Das Pop-Up mit der Erinnerung erscheint auf dem Smartwatch-Display
\item Der Patient wählt Option zum Verschieben der Medikation aus
\item Auf einem zusätzlichem Dialog kann er aus einer Auswahl eine Zeitdauer wählen, um das die Medikation verschoben wird}
\additemizedfield{Alternativablauf 2}{
\item Das Pop-Up mit der Erinnerung erscheint auf dem Smartwatch-Display
\item Der Patient reagiert nicht auf die Erinnerung
\item Die Erinnerung wird alle x Minuten wiederholt, solange der Patient nicht reagiert.
\item Das Fehlen einer Reaktion des Patientens innerhalb einer Zeit von x Minuten wird vermerkt}
\addfield{Ergebnis}{Der Patient hat die Einnahme des Medikaments bestätigt und dieses auch eingenommen}
\addfield{Alternativergebnis 1}{Der Patient hat die Erinnerung an die Medikamenteneinnahme verschoben}
\addfield{Alternativergebnis 2}{Der Patient hat die Erinnerung an das Medikament ausgeschaltet}
\end{usecase}

\begin{usecase}
\addtitle{Usecase 2}{Der Patient wird an mehrere Medikamente erinnert}
\addfield{Primärer Akteur}{Patient}
\addfield{Beschreibung}{Ein Erinnerungs Pop-Up erscheint auf dem Display und es wird ein Signal/eine Vibration ausgelöst. Das Pop-Up zeigt eine Liste der Medikamente, die eingenommen werden müssen.}
\addfield{Vorbedingung}{Es wurde ein Medikationsplan aus der DB auf die Uhr geladen.}
\additemizedfield{Ablauf}{
\item Das Pop-Up mit der Erinnerung erscheint auf dem Smartwatch-Display. Die Erinnerung beinhaltet Informationen zur Uhrzeit, Menge und Art der Medikationen
\item Der Patient bestätigt, dass er die Medikamente alle genommen hat
\item Gerät bestätigt visuell, dass der Patient die Medikamente als genommen markiert hat
}
\additemizedfield{Alternativablauf}{
\item Das Pop-Up mit der Erinnerung erscheint auf dem Smartwatch-Display
\item Der Patient wählt Option zum Verschieben der Medikationen aus
\item Auf einem zusätzlichem Dialog kann er aus einer Auswahl eine Zeitdauer wählen, um das die Medikationen verschoben werden}
\additemizedfield{Alternativablauf 2}{
\item Das Pop-Up mit der Erinnerung erscheint auf dem Smartwatch-Display
\item Der Patient wählt ein Medikament von der Liste aus
\item Patient befindet sich nun im Usecase 1}
\additemizedfield{Alternativablauf 3}{
\item Das Pop-Up mit der Erinnerung erscheint auf dem Smartwatch-Display
\item Der Patient reagiert nicht auf die Erinnerung
\item Die Erinnerung wird alle x Minuten wiederholt, solange der Patient nicht reagiert.
\item Das Fehlen einer Reaktion des Patientens innerhalb einer Zeit von x Minuten wird vermerkt}
\addfield{Ergebnis}{Der Patient hat die Einnahme der MEdikamente bestätigt und dieses auch eingenommen}
\addfield{Alternativergebnis 1}{Der Patient hat die Erinnerung an die Medikamenteneinnahme verschoben}
\addfield{Alternativergebnis 2}{Der Patient hat die Erinnerung an das Medikament ausgeschaltet}
\addfield{Alternativergebnis 3}{Der Patient hat bestimmte Medikamente ausgewählt und mit dem weiterführenden Usecase 1 bearbeitet}
\end{usecase}

\begin{usecase}
\addtitle{Usecase 3}{Einzelne Einnahmebestätigung zurücknehmen}
\addfield{Primärer Akteur}{Patient}
\addfield{Vorbedingung}{Es wurde ein Medikationsplan aus der DB auf die Uhr geladen. Eine Medikation wurde als genommen makiert}
\additemizedfield{Ablauf}{
\item Das System zeigt das genommene Medikament an
\item Der Benutzer drückt auf den Button mit der Aufschrift “Rücknahme”
\item Das System wechselt zur Darstellung eines einzelnen Medikaments, beschrieben im UseCase 1
}
\addfield{Ergebnis}{Die Einnahmebestätigung ist zurückgenommen. Die Erinnerung ist erneut zu bestätigen oder zu verschieben.}
\end{usecase}

\begin{usecase}
\addtitle{Usecase 4}{Mehrere Einnahmebestätigungen zurücknehmen}
\addfield{Primärer Akteur}{Patient}
\addfield{Vorbedingung}{Es wurde ein Medikationsplan aus der DB auf die Uhr geladen. Mehrere Medikationen, welche zur gleichen Zeit genommen wurden, wurde als genommen makiert}
\additemizedfield{Ablauf}{
\item Das System zeigt die genommenen Medikamente an
\item Der Benutzer drückt auf den Button mit der Aufschrift “Rücknahme”
\item Das System wechselt zur Darstellung mehrere Medikamente, beschrieben im UseCase 2
}
\addfield{Ergebnis}{Die Einnahmebestätigung ist zurückgenommen. Die Erinnerung ist erneut zu bestätigen oder zu verschieben.}
\end{usecase}

