%%
% ----------------------------------------------------------------------------
% "THE BEER-WARE LICENSE" (Revision 42):
% <sebastian.rauh@hs-heilbronn.de/michael.bauer@hs-heilbronn.de> wrote this 
% file. As long as you retain this notice you can do whatever you want with 
% this stuff. If we meet some day, and you think this stuff is worth it, you 
% can buy us a beer in return. 
% Michael Lukas Bauer, Sebastian Felix Rauh
% ----------------------------------------------------------------------------
%%

%%
% ----------------------------------------------------------------------------
% "THE BEER-WARE LICENSE" (Revision 42):
% <sebastian.rauh@hs-heilbronn.de/michael.bauer@hs-heilbronn.de> wrote this 
% file. As long as you retain this notice you can do whatever you want with 
% this stuff. If we meet some day, and you think this stuff is worth it, you 
% can buy us a beer in return. 
% Michael Lukas Bauer, Sebastian Felix Rauh
% ----------------------------------------------------------------------------
%%

Während das Forschungsfeld der \glqq Medication Adherence\grqq - das Einhalten des Medikamentenplans - schon jahrzehntelang erforscht wurde, ist diese Arbeit Teil eines noch sehr jungen Forschungsgebietes. Smartwatches existieren noch nicht lange im Consumer Bereich. Im folgenden ist eine Zusammenfassung der relevanten Forschungen zu finden. Apple`s Programmiersprache Swift sowie die Technologie der Wearables wird beschrieben.
\section{Arbeiten im Forschungsumfeld}
Leider gibt es kaum Arbeiten, welche sich mit dem Thema Smartwatch und \glqq Medication Adherence\grqq   beschäftigen. Dies ist auf das noch junge Forschungsfeld zurückzuführen. Sailer`s Artikel \cite{Fabian-Sailer:2015aa} bietet einen Einstieg für diese Arbeit. Hier wurde auch die Thematik für das \gls{pita} abgeleitet, dessen Erkenntnisse hier fortgeführt werden. 

Weiter gibt es sehr spannende Forschungen im Bereich der Smartwatchanwendung, die in fortschreitender Entwicklung auch im Bereich \glqq Medication Adherence\grqq vorstellbar sind. Mit Ambient Assisted Living, also der technischen Unterstützung älterer oder eingeschränkter Personen im Haushalt, beschäftigt sich die Arbeit \glqq Non-obstructive Room-level Locating System in Home Environments Using Activity Fingerprints from Smartwatch\grqq\cite{Lee:2015:NRL:2750858.2804272}. Die von dieser Arbeit abgeleitete Lösung kann auch für rechtzeitige Medikamenteneinnahme genutzt werden. So könnte ein Medikationsalarm nur ausgelöst werden, wenn der Patient sich auch im richtigen Zimmer befindet. Durch den kürzeren Weg zum Medikament sinkt die Gefahr, auf der Suche nach dem Medikament, die Einnahme wieder zu vergessen. Die Arbeit von Laput gliedert sich ebenfalls in diesen Bereich ein und hat auch einen Kontextgewinn zur Folge \cite{Laput:2015:ETR:2807442.2807481}. Berührt der Träger des  Smartwatch Prototyps einen Gegenstand, so erkennt die Uhr das elektromagnetische Feld des Gegenstandes. Durch maschinelles Lernen werden nun die Gegenstände mit ihrem elektromagnetischen Feld verknüpft. Nun kann die Uhr erkennen, welchen Gegenstand der Träger berührt. Dieser Kontextgewinn, welche Gegenstände der Träger der Uhr berührt, könnte Fehler bei Medikamenteneinnahmen verhindern, indem die Uhr die Medikamente erkennt, die der Patient berührt.
Die Studie \glqq Smartwatch in vivo\grqq \cite{Pizza:2016} untersucht das Nutzungsverhalten von Smartwatch Nutzern. Hierzu trägt der Nutzer eine Schulterkamera, die die Interaktion mit der Uhr über drei Tage filmt. So zeigt die Studie auf, dass neben der Uhrzeit (mit ca. 50\% Nutzungsdauer) die Notification mit 20\% an Nutzungsdauer die häufigste Interaktion ist. Anwendungen werden so gut wie nie genutzt. Auch die durchschnittliche Interaktionszeit von ca. 7s ist ein wichtiges Ergebnis.

Orientiert man sich an Arbeiten, deren Ziel die smartphonegestützte Medication Adherence war, findet man unter anderem die aktuelle nationale Umfrage \cite{Krebs-P:2015aa} aus den USA, bei der 1604 Smartphone Nutzer zu ihrer Nutzung von Gesundheitsanwendungen befragt wurden. Fast 50\% der Befragten gaben an, eine oder mehrere Gesundheitsanwendungen auf ihrem Smartphone installiert zu haben. Eine wichtige Aussage der Umfrage war, dass ein Großteil der Nutzer nicht bereit ist, für Gesundheitsanwendungen zu bezahlen.

