%%
% ----------------------------------------------------------------------------
% "THE BEER-WARE LICENSE" (Revision 42):
% <sebastian.rauh@hs-heilbronn.de/michael.bauer@hs-heilbronn.de> wrote this 
% file. As long as you retain this notice you can do whatever you want with 
% this stuff. If we meet some day, and you think this stuff is worth it, you 
% can buy us a beer in return. 
% Michael Lukas Bauer, Sebastian Felix Rauh
% ----------------------------------------------------------------------------
%%

\documentclass[
	12pt,
	BCOR=10mm,
	listof=totoc,
	bibliography=totoc
	]
{scrbook} %BCOR = Binderandkorrekur, in Druckerrei vor Druck nachfragen wie breit
%\usepackage{mathptmx}  - sets \rmdefault to 'ptm', i.e. times
\usepackage[utf8]{inputenc}
\usepackage{parskip}
\usepackage[english,ngerman]{babel}
\usepackage{amsmath,amssymb,amstext}
\usepackage{graphicx}
\usepackage{tabularx}
\usepackage{fancyhdr}
\usepackage[bookmarksopenlevel=section]{hyperref}
\usepackage[toc]{glossaries}
\usepackage{cite}
\usepackage{multicol}
\usepackage{listings}
\usepackage[svgnames]{xcolor}
%Imported by Lukas Schmidt 28.10.15
\usepackage{usecases}
\usepackage{listings}
\usepackage[colorinlistoftodos,prependcaption,textsize=tiny]{todonotes}
\lstdefinestyle{base}{
columns=fullflexible,
  emptylines=1,
  breaklines=true,
  basicstyle=\ttfamily\color{black},
  moredelim=**[is][\color{red}]{@}{@},
}
\usepackage{url}
\lstset{style=base, frame=shadowbox}
\usepackage{tasks}
% Globals
\newcommand{\autor}{Lukas Schmidt}
\newcommand{\titel}{System zur Erinnerung an Medikamenteneinnahmen für die Apple Watch mit Blick auf Gebrauchstauglichkeit und Akzeptanz  der Plattform}
\newcommand{\supervisor}{Prof. Dr.-Ing. Gerrit Meixner}
\newcommand{\reviewer}{Prof. Dr. Martin Haag}
\newcommand{\matriculationNo}{182706}

% Definition XML als Sprache
\lstdefinestyle{XML} {
    language=XML,
    extendedchars=true,
    %xleftmargin=21pt,
    frame=single,
    %framexleftmargin=20pt,
    numbers=left,
    numberstyle=\footnotesize\ttfamily\color{Black}, 
    breaklines=true,
    breakatwhitespace=true,
    emph={},
    emphstyle=\color{Red},
    basicstyle=\ttfamily\color{Black},
    columns=fullflexible,
    commentstyle=\color{Green}\upshape,
    morestring=[b]",
    morecomment=[s]{<?}{?>},
    morecomment=[s][\color{Green}]{<!--}{-->},
    keywordstyle=\color{Red},
    stringstyle=\ttfamily\color{Blue}\normalfont,
    tagstyle=\color{Maroon}\bf,
    morekeywords={attribute,xmlns,version,type,release,name,use,elementFormDefault,attributeFormDefault,maxOccurs},
    otherkeywords={xmlns:xs},
}

%Test
\definecolor{rosso}{RGB}{220,57,18}
\definecolor{giallo}{RGB}{255,153,0}
\definecolor{blu}{RGB}{102,140,217}
\definecolor{verde}{RGB}{16,150,24}
\definecolor{viola}{RGB}{153,0,153}

\makeatletter

\tikzstyle{chart}=[
    legend label/.style={font={\scriptsize},anchor=west,align=left},
    legend box/.style={rectangle, draw, minimum size=5pt},
    axis/.style={black,semithick,->},
    axis label/.style={anchor=east,font={\tiny}},
]

\tikzstyle{bar chart}=[
    chart,
    bar width/.code={
        \pgfmathparse{##1/2}
        \global\let\bar@w\pgfmathresult
    },
    bar/.style={very thick, draw=white},
    bar label/.style={font={\bf\small},anchor=north},
    bar value/.style={font={\footnotesize}},
    bar width=.75,
]

\tikzstyle{pie chart}=[
    chart,
    slice/.style={line cap=round, line join=round, very thick,draw=white},
    pie title/.style={font={\bf}},
    slice type/.style 2 args={
        ##1/.style={fill=##2},
        values of ##1/.style={}
    }
]

\pgfdeclarelayer{background}
\pgfdeclarelayer{foreground}
\pgfsetlayers{background,main,foreground}


\newcommand{\pie}[3][]{
    \begin{scope}[#1]
    \pgfmathsetmacro{\curA}{90}
    \pgfmathsetmacro{\r}{1}
    \def\c{(0,0)}
    \node[pie title] at (90:1.3) {#2};
    \foreach \v/\s in{#3}{
        \pgfmathsetmacro{\deltaA}{\v/100*360}
        \pgfmathsetmacro{\nextA}{\curA + \deltaA}
        \pgfmathsetmacro{\midA}{(\curA+\nextA)/2}

        \path[slice,\s] \c
            -- +(\curA:\r)
            arc (\curA:\nextA:\r)
            -- cycle;
        \pgfmathsetmacro{\d}{max((\deltaA * -(.5/50) + 1) , .5)}

        \begin{pgfonlayer}{foreground}
        \path \c -- node[pos=\d,pie values,values of \s]{$\v\%$} +(\midA:\r);
        \end{pgfonlayer}

        \global\let\curA\nextA
    }
    \end{scope}
}

\newcommand{\legend}[2][]{
    \begin{scope}[#1]
    \path
        \foreach \n/\s in {#2}
            {
                  ++(0,-10pt) node[\s,legend box] {} +(5pt,0) node[legend label] {\n}
            }
    ;
    \end{scope}
}
%test End

% Header and Footer
\pagestyle{fancy}
\fancyhf{} % clear the headers
\fancyhead[LE,RO]{%
   % We want italics
   \itshape   
   \leftmark
   % The chapter title
   }
\fancyfoot[LE,RO]{\thepage}
\renewcommand{\chaptermark}[1]{\markboth{#1}{}}

% Font Times New Roman
\renewcommand{\familydefault}{ptm} % using \rmdefault here doesn't work

% PDF Property
\hypersetup{
	pdftitle={\titel},
	pdfsubject={},
	pdfauthor={\autor},
	pdfkeywords={},
}

% Bibliography
\bibliographystyle{acm}

% Glossary
\makeglossaries
\glossarystyle{list}
%%
% ----------------------------------------------------------------------------
% "THE BEER-WARE LICENSE" (Revision 42):
% <sebastian.rauh@hs-heilbronn.de/michael.bauer@hs-heilbronn.de> wrote this 
% file. As long as you retain this notice you can do whatever you want with 
% this stuff. If we meet some day, and you think this stuff is worth it, you 
% can buy us a beer in return. 
% Michael Lukas Bauer, Sebastian Felix Rauh
% ----------------------------------------------------------------------------
%%

% Ultraschall Funktionen
\newglossaryentry{pdc}{
	type=\acronymtype, 
	name=PDC,
	first=Park Distance Control/Park Pilot (PDC),
	description=Warning for obstacles in front and behind the car
}
\newglossaryentry{sdw}{
	type=\acronymtype, 
	name=SDW,
	first=Side Distance Warning (SDW),
	description=Warning for obstacles in beside the car
}
\newglossaryentry{nra}{
	type=\acronymtype, 
	name=NRA,
	first=Narrow Road Assist (NRA),
	description=Visualising PDC and SDW by taking the angle of steering lock respectively the driving tube into account
}
\newglossaryentry{psc}{
	type=\acronymtype, 
	name=PSC,
	first=Park Steer Control (PSC),
	description=Active steering into longitudinal parking spots
}
\newglossaryentry{cpsc}{
	type=\acronymtype, 
	name=CPSC,
	first=Cross Park Steer Control (CPSC),
	description=Active steering into crosswise parking spots
}
\newglossaryentry{parking spot measurement}{
	name=Parking Spot Measurement,
	description=Measurement and visualisation during the passing of the parking spot
}
\newglossaryentry{poc}{
	type=\acronymtype, 
	name=POC
	first=Pull-Out Control (POC),
	description=Active steering during pull out of the car
}
\newglossaryentry{ps}{
	type=\acronymtype, 
	name=PS,
	first=Park Stop (PS),
	description=Regulation of speed by actively longitudinal plug breaking as an addition Feature to all Features except parking spot measurement
}
\newglossaryentry{pba}{
	type=\acronymtype, 
	name=PBA,
	first=Park Brake Assist (PBA),
	description=See Park Stop (PS)
}
\newglossaryentry{sva}{
	type=\acronymtype, 
	name=SVA,
	first=Side View Assist (SVA),
	description=Observation of dead corners includes road work assist
}
\newglossaryentry{sdi}{
	type=\acronymtype,
	name=SDI,
	first=Side Distance Assist (SDA),
	description=See Side View Assist
}

% Informatische Fachbegriffe
\newglossaryentry{xml}{
	type=\acronymtype, 
	name=XML, 
	description={A markup language that defines a set of rules for encoding documents in a format that is both human-readable and machine-readable. It is defined in the XML 1.0 Specification produced by the W3C, and several other related specifications, all free open standards}, 
	first=Extensible Markup Language (XML)
}
\begin{document}



% Sprache
\selectlanguage{ngerman}

\begin{titlepage}
	%%
% ----------------------------------------------------------------------------
% "THE BEER-WARE LICENSE" (Revision 42):
% <sebastian.rauh@hs-heilbronn.de/michael.bauer@hs-heilbronn.de> wrote this 
% file. As long as you retain this notice you can do whatever you want with 
% this stuff. If we meet some day, and you think this stuff is worth it, you 
% can buy us a beer in return. 
% Michael Lukas Bauer, Sebastian Felix Rauh
% ----------------------------------------------------------------------------
%%

\begin{center}
\hfill
\begin{minipage}{0.45\textwidth}
\begin{flushleft}
\includegraphics[width=0.9\textwidth]{00_title/pics/UniHD} \\
\end{flushleft}
\end{minipage}
\hfill
\begin{minipage}{0.45\textwidth}
\begin{flushright}
\includegraphics[width=0.9\textwidth]{00_title/pics/HHN} \\
\end{flushright}
\end{minipage}
\hfill \\[3.0cm]

\textsc{\huge \bfseries Bachelor-Thesis}\\[1.0cm]
%\textsc{\huge \bfseries Arbeit}\\[3.0cm]

% Title
{ \Large \bfseries \titel}\\[6.5cm]

% Tabelle
\def\arraystretch{1.2}
\begin{tabularx}{\columnwidth}{|ll|X|}
\hline
Autor & \quad \quad & \autor \\
\hline
Studiengang &  & Medizinische Informatik \\
 & & Universtität Heidelberg / Hochschule Heilbronn\\
\hline
Matrikelnummer &   & \matriculationNo \\
\hline
Abgabe &  & 31. März 2016 \\
\hline
Referent &  & \supervisor \\
\hline
Korreferent &  & \reviewer \\
\hline
\end{tabularx}
\vfill

\end{center}
\end{titlepage} 

\frontmatter
\section*{Zusammenfassung}


Im fortschreitenden Alter fällt es Menschen schwer, regelmäßig Medikamente zum richtigen Zeitpunkt einzunehmen. Oft wird dies durch eine große Anzahl verschiedener Medikamente noch erschwert, die über den Tag eingenommen werden müssen. 

In Zusammenarbeit mit dem \gls{pita} an der HS-Heilbronn wird eine Smartwatch-Anwendung für die Apple Watch entwickelt. Diese basiert stark auf dezenten Benachrichtigungen am Handgelenk. Auch eine native Anwendung, die der Nutzer aktiv bedienen kann, wird erstellt.

Die Anwendung ist aufgrund fehlender Prototypen-Wekzeuge für die Apple Watch nativ in Swift realisiert. Hierbei wird auf Swift als relativ neue Programmiersprache eingegangen.

Die Evaluierung wird an stationären Patienten im Alter von 70-85 vorgenommen. Die Aussage der Befragung ergibt, dass sich eine Uhr als Medium sehr gut eignet, da sie etwas Vertrautes ausstrahlt. Die touchscreen-basierte Steuerung fällt aufgrund reduzierter sensomotorischer Fähigkeiten der Probanden negativ auf. Während die Benachrichtigung mit einhergehender Vibration sehr gut aufgenommen wird, ist die Interaktion mit der Uhr schwerfällig. Die Patienten haben Probleme eine native Anwendung zu starten. In Zukunft könnten diese Probleme mit Hilfe von Accessibility Funktionen der Plattform gelöst werden. Weiter bietet die Uhr interessante Anwendungsmöglichkeiten Menschen im Alltag zu unterstützen.

\tableofcontents

\listoffigures

\listoftables


\twocolumn
\printglossaries

\onecolumn
\mainmatter
\chapter{Einleitung}
\label{ch:introduction}
 	\section{Motivation}
Durch steigende Anzahl an Medikamenten, die Patienten über den Tag nehmen müssen, kann dies zu einer großen mentalen Aufgabe für den Patienten werden. Durch Hilfsmittel wie Medikationspläne oder nach Zeit vorsortierte Medikamente, kann die Planung der Einnahme unterstützt werden. 

Zur Erleichterung der Patienten soll der Medikationsplan nun per Smartwatch einsehbar gemacht werden und somit dem Patienten eine Interaktion mit dem Plan ermöglichen. Daraus bieten sich auch Vorteile für den behandelnden Arzt, der Einblick in die Einnahmegewohnheiten seines Patienten bekommt. Die initialen Ideen stammen vom \gls{pita} an der HS-Heilbronn. An der HS-Heilbronn wird eine Komponente für Ärzte entwickelt, die es ihnen ermöglicht, die Medikationen der Patienten zu pflegen, zu überwachen und auszuwerten.
\section{Zielsetzung}
Die Ziele der Arbeit sind wie folgt.

Es wird ein interaktiver Prototyp entwickelt, welcher auf der Apple Watch ausgeführt werden kann. Dieser Prototyp soll dann mit geeigneten Probanden, die zur passenden Zielgruppe gehören, evaluiert werden.
Die Ergebnisse der Evaluierung sollen im Rahmen der technischen Möglichkeiten umgesetzt werden. Zum Schluss soll noch die Machbarkeit überprüft werden, die Anwendung an das Backend-System vom \gls{pita} anzuschließen.

\section{Aufbau der Arbeit}

In Kapitel 2 wird auf das Forschungsumfeld der Arbeit Bezug genommen. Weiter werden technologische Grundlagen beschrieben wie Apple`s Swift Programmiersprache und die Technologie von Wearables.
Im dritten Kapitel werden Anforderungen aufgeführt, die teilweise aus dem  PITA-Praktikum stammen. Weiter werden Anforderungen geschildert, die sich aus den Möglichkeiten der Apple Watch ergeben.
Abschnitt 4 betrachtet die Kernpunkte der Implementierung. Weiter wird hier der Prototyp beschrieben.
In Kapitel 5 wird die Planung und Durchführung der Evaluierung dargestellt. Die  Ergebnisse der Evaluierung an der Zielgruppe sind hier ebenfalls zu finden.
In der Diskussion in Kapitel 6 wird aufgezeigt, welche Probleme während der Arbeit entstanden sind. Es wird veranschaulicht, wie der Prototyp im Gesamtbild einzuordnen ist und es wird ein Ausblick gegeben, welche Ziele mit dem System weiter verfolgt werden können.


  
\chapter{Stand der Wissenschaft}
\label{ch:state-art}
 	%%
% ----------------------------------------------------------------------------
% "THE BEER-WARE LICENSE" (Revision 42):
% <sebastian.rauh@hs-heilbronn.de/michael.bauer@hs-heilbronn.de> wrote this 
% file. As long as you retain this notice you can do whatever you want with 
% this stuff. If we meet some day, and you think this stuff is worth it, you 
% can buy us a beer in return. 
% Michael Lukas Bauer, Sebastian Felix Rauh
% ----------------------------------------------------------------------------
%%

%%
% ----------------------------------------------------------------------------
% "THE BEER-WARE LICENSE" (Revision 42):
% <sebastian.rauh@hs-heilbronn.de/michael.bauer@hs-heilbronn.de> wrote this 
% file. As long as you retain this notice you can do whatever you want with 
% this stuff. If we meet some day, and you think this stuff is worth it, you 
% can buy us a beer in return. 
% Michael Lukas Bauer, Sebastian Felix Rauh
% ----------------------------------------------------------------------------
%%

\section{Abstract}
Im fortschreitenden Alter fällt es Menschen schwer, regelmäßig Medikamente zum richtigen Zeitpunkt einzunehmen. Oft wird dies durch eine große Anzahl verschiedener Medikamente noch erschwert, die über den Tag eingenommen werden müssen. 

In Zusammenarbeit mit dem \gls{pita} an der HS-Heilbronn wird eine Smartwatch-Anwendung für die Apple Watch entwickelt. Diese basiert stark auf dezenten Benachrichtigungen am Handgelenk. Auch eine native Anwendung, die der Nutzer aktiv bedienen kann, wird erstellt.

Die Anwendung ist aufgrund fehlender Prototypen-Wekzeuge für die Apple Watch nativ in Swift realisiert. Hierbei wird auf Swift als relativ neue Programmiersprache eingegangen.

Die Evaluierung wird an stationären Patienten im Alter von 70-85 vorgenommen. Die Aussage der Befragung ergibt, dass sich eine Uhr als Medium sehr gut eignet, da sie etwas Vertrautes ausstrahlt. Die touchscreen-basierte Steuerung fällt aufgrund reduzierter sensomotorischer Fähigkeiten der Probanden negativ auf. Während die Benachrichtigung mit einhergehender Vibration sehr gut aufgenommen wird, ist die Interaktion mit der Uhr schwerfällig. Die Patienten haben Probleme eine native Anwendung zu starten. In Zukunft könnten diese Probleme mit Hilfe von Accessibility Funktionen der Plattform gelöst werden. Weiter bietet die Uhr interessante Anwendungsmöglichkeiten Menschen im Alltag zu unterstützen.

\section{Einleitende Worte}
Während das Forschungsfeld der \glqq Medication Adherence\grqq - das Einhalten des Medikamentenplans - schon jahrzehntelang erforscht wurde, ist diese Arbeit Teil eines noch sehr jungen Forschungsgebietes. Smartwatches existieren noch nicht lange im Consumer Bereich. Im folgenden ist eine Zusammenfassung der relevanten Forschungen zu finden. Apple`s Programmiersprache Swift sowie die Technologie der Wearables wird beschrieben.
\section{Arbeiten im Forschungsumfeld}
Leider gibt es kaum Arbeiten, welche sich mit dem Thema Smartwatch und \glqq Medication Adherence\grqq   beschäftigen. Dies ist auf das noch junge Forschungsfeld zurückzuführen. Sailer`s Arbeit \cite{Fabian-Sailer:2015aa} bietet einen Einstieg für diese Arbeit. Hier wurde auch die Thematik für das \gls{pita} abgeleitet, dessen Erkenntnisse hier fortgeführt werden. 

Weiter gibt es sehr spannende Forschungen im Bereich der Smartwatchanwendung, die in fortschreitender Entwicklung auch im Bereich \glqq Medication Adherence\grqq vorstellbar sind. Mit Ambient Assisted Living, also der technischen Unterstützung älterer oder eingeschränkter Personen im Haushalt, beschäftigt sich die Arbeit \glqq Non-obstructive Room-level Locating System in Home Environments Using Activity Fingerprints from Smartwatch\grqq\cite{Lee:2015:NRL:2750858.2804272}. Die von dieser Arbeit abgeleitete Lösung kann auch für rechtzeitige Medikamenteneinnahme genutzt werden. So könnte ein Medikationsalarm nur ausgelöst werden, wenn der Patient sich auch im richtigen Zimmer befindet. Durch den kürzeren Weg zum Medikament sinkt die Gefahr, auf der Suche nach dem Medikament, die Einnahme wieder zu vergessen. Die Arbeit von Laput gliedert sich ebenfalls in diesen Bereich ein und hat auch einen Kontextgewinn zur Folge \cite{Laput:2015:ETR:2807442.2807481}. Berührt der Träger des  Smartwatch Prototyps einen Gegenstand, so erkennt die Uhr das elektromagnetische Feld des Gegenstandes. Durch maschinelles Lernen werden nun die Gegenstände mit ihrem elektromagnetischem Feld verknüpft. Nun kann die Uhr erkennen, welchen Gegenstand der Träger berührt. Dieser Kontextgewinn, welche Gegenstände der Träger der Uhr berührt, könnte Fehler bei Medikamenteneinnahmen verhindern, indem die Uhr die Medikamente erkennt, die der Patient berührt.
Die Studie \glqq Smartwatch in vivo\grqq \cite{Pizza:2016} untersucht das Nutzungsverhalten von Smartwatch Nutzern. Hierzu trägt der Nutzer eine Schulterkamera, die die Interaktion mit der Uhr über drei Tage filmt. So zeigt die Studie auf, dass neben der Uhrzeit (mit ca. 50\% Nutzungsdauer) die Notification mit 20\% an Nutzungsdauer die häufigste Interaktion ist. Anwendungen werden so gut wie nie genutzt. Auch die durchschnittliche Interaktionszeit von ca. 7s ist ein wichtiges Ergebnis.

Orientiert man sich an Arbeiten, deren Ziel die smartphonegestützte Medication Adherence war, findet man unter anderem die aktuelle nationale Umfrage \cite{Krebs-P:2015aa} aus den USA, bei der 1604 Smartphone Nutzer zu ihrer Nutzung von Gesundheitsanwendungen befragt wurden. Fast 50\% der Befragten gaben an, eine oder mehrere Gesundheitsanwendungen auf ihrem Smartphone installiert zu haben. Eine wichtige Aussage der Umfrage war, dass ein Großteil der Nutzer nicht bereit ist für Gesundheitsanwendungen zu bezahlen.

\section{Apple`s Programmiersprache Swift}
Swift wurde im Juni 2014 von Apple vorgestellt und genießt seitdem steigendes Interesse. Im Juni 2015 wurde Version 2.0 veröffentlicht \cite{Apple:2014sp}. Mit Version 2.0 wurde ebenfalls der Plan vorgestellt, Swift Open Source zu machen und somit auch anderen Plattformen die Entwicklung mit Swift zu ermöglichen \cite{Apple:2014sp}. Diesen Plan setzte Apple Ende 2015 in die Tat um. Swift ist nun völlig Open Source. Es können Vorschläge für neue Sprachfunktionen gemacht werden. Auch Apple`s Entwicklungsteam diskutiert seine Pläne für die Sprache öffentlich \cite{Apple:2015swiftOpen}.

\subsection{Objective-C und Swift}
In den frühen 80er Jahren entwickelte Brad Cox die Sprache Objective-C\cite{Dalrymple:2009aa}. Weiter führt Dalrymple aus, dass die Sprache  die Vorteile einer schnellen C-Sprache mit den Vorteilen der objektorientierten Sprache SmallTalk verbinden sollte. Die Firma NextSTEP nutzte Objectiv-C und als NEXTStep von Apple aufgekauft wurde, integrierte Apple Objectiv-C und ermöglichte Mac-Entwicklern die Nutzung.

 Als Apple nun 2008 seine iOS Plattform öffnete und Entwickler eigene Anwendungen für das System schreiben konnten, bekam Objective-C neue Aufmerksamkeit. Viele Entwickler sahen Objective-C als ein Überbleibsel alter Zeiten und waren der Sprache gegenüber negativ eingestellt. Apple stand nun unter Zugzwang um seine Plattform, mit der große wirtschaftliche Interessen verbunden sind, für Entwickler attraktiv zu halten \cite{Wells:2015fu}. Da jedoch alle highlevel APIs in Objectiv-C vorhanden sind, ist es technisch nicht möglich, auf eine bekannte Sprache wie Java für iOS und Mac Entwicklung umzusteigen. Man entschied sich für eine Neuentwicklung mit Hinblick auf neue Programmierparadigmen und auf sehr gute Kompatibilität zu alten Objectiv-C APIs \cite{Wells:2015fu}.

\subsection{Überblick der Neuerung}
\begin{enumerate}
\item Type Inference
\item Closures - Functions as First Class Types
\item Generics
\item Optional Types
\end{enumerate}

\subsection{Type Inference}
Bei Type Inference erkennt der Compiler, welcher Typ in eine Variable instanziiert wurde. Es ist nicht nötig für die Variable eine Typendefinition anzugeben \cite{Apple:2014sp}. Hierdurch wird der Quellcode leichter lesbar. 
\lstinputlisting[caption=Beispiel zu Type Inference in Swift label=lst:typeInference]{02_stateArt/codeExamples/TypeInference.swift}

\subsection{Closures - Functions as First Class Types}
\label{ch:closures}
Während bei strikt objektorientierten Sprachen nur Objekte und primitive Datentypen existieren, gibt es Sprachen, bei denen Funktionen als Typen existieren. Diese Funktionen können auch als Referenz in eine Variable gespeichert werden. Folgende Code Beispiele sollen dies veranschaulichen. Wir nutzen hierfür einen asynchronen Netzwerk-Request\cite{Apple:2014sp}.
\lstinputlisting[caption=Java Beispiel zu First Class Objects, label=lst:firstFuncJava]{02_stateArt/codeExamples/NetworkRequestWithoutFunctions.java}
 In Beispiel \ref{lst:firstFuncJava} wird Java als Repräsentant für eine strikt objektorientierte Sprache verwendet. Hier wird ein Interface definiert, welches der Nutzer des Netzwerk-Requests implementieren muss, um den Request zu empfangen. Beim Aufrufen des Requests muss nun der Aufrufer als Referenz übergeben werden, damit bei Abschließen des Requests der Aufrufer benachrichtigt werden kann (handleNetworkResponse). 
\lstinputlisting[caption=Swift Beispiel zu First Class Objects, label=lst:firstFuncSwift]{02_stateArt/codeExamples/NetworkRequestWithFunctions.swift}
Im Codebeispiel \ref{lst:firstFuncSwift} benötigt es kein Interface für den Nutzer des Netzwerk-Requests. Es ist nun in Swift möglich, eine Funktion zu definieren und diese gleichzeitig in einer lokalen Variablen zu speichern. Nun kann diese Funktion als Referenz zum Netzwerk-Request übergeben werden. Die Funktion wird aufgerufen, wenn der Netzwerk-Request beendet ist. 

\subsection{Generics}
Generics sind schon längere Zeit Teil moderner Programmiersprachen. Sie unterstützen den Entwickler, um Fehler zur Compilezeit zu entdecken. Ein sehr gutes Beispiel hierfür sind Collections (Arrays, Listen, Set, etc). Ein Collection Typ wie Array muss nicht für jeden Typ, den er hält, neu implementiert werden.
\lstinputlisting[caption=Swift Beispiel zu Generic Array Collection, label=lst:generics]{02_stateArt/codeExamples/Generics.swift}
Im Codebeispiel \ref{lst:generics} werden die Vorteile von Generics aufgezeigt. Der erste Versuch einen Array zu implementieren (CustomMedicationArray) zeigt, dass dieser Array eine sehr gute Typen Deklarierung enthält. Der Compiler kann den Entwickler also maximal unterstützen, jedoch ist diese Implementierung minimal wiederverwendbar. Der zweite Ansatz (CustomArray) gibt als Typ-Einschränkung den globalen Supertyp an. Dies führt zu einer maximalen Wiederverwendbarkeit, da jede Klasse von diesem globalen Supertyp erbt (funktioniert nur theoretisch, da in Swift kein Zwang besteht, von einem globalen Supertyp zu erben).
Die letzte Implementierung (GenericArray) nutzt nun Generics. So wird bei der Definition der Klasse ein generischer Typ \glqq T\grqq  eingeführt. Dieser Typ ist eine Art Platzhalter für einen konkreten Typ, der später vom Array gehalten wird. So muss bei der Initialisierung der generischen Klasse die Typ-Information für T übergeben werden (siehe letze Zeile in \ref{lst:generics}).

Das oben beschriebene Beispiel dient nur zur Verdeutlichung des Konzeptes. Swift bietet eine Reihe an Collection Types, darunter auch eine generische Array Implementierung. 

\subsection{Protocol Extension}
Protocol Extension helfen dem Entwickler Abstraktionen, die über ein Interface (in Swift Protocol) definiert werden, zu implementieren und dadurch Duplizierungen zu minimieren. So kann es sein, dass eine Methode eines Interfaces in jeder Klasse, die das Interface implementiert, gleich umgesetzt wird. Dies führt zu einer Code-Duplizierung. 
\lstinputlisting[caption=Swift Beispiel zu Generic Array Collection, label=lst:ProtocolExtensions]{02_stateArt/codeExamples/ProtocolExtensions.swift}
Im Codebeispiel \ref{lst:ProtocolExtensions} wird ein List Interface definiert. Die Extension zu diesem Interface ermöglicht die Implementierung von \glqq last\grqq  und \glqq first\grqq. Diese Implementierung teilen sich nun alle Klassen, die List implementieren.

\subsection{Optional Types}
\label{ch:optionals}
Optional Types eröffnen neue Möglichkeiten beim Modellieren von Datenmodellen und Erstellen von APIs. Es ist so möglich, Argumente in einer Methode als \glqq 0ptional\grqq  zu kennzeichnen und somit dem Nutzer der Methode zu erlauben, eine null-Referenz zu übergeben. Ist kein Optional-Type gekennzeichnet, so verbietet der Compiler eine null-Referenz Übergabe \cite{Apple:2014sp}. Im folgenden Beispiel wird eine Medikation in Java ohne Optional Types und in Swift mit Optional Types modelliert.
\lstinputlisting[caption=Java Beispiel mit fehlenden Optional Types, label=lst:optinalsJava]{02_stateArt/codeExamples/OptionalType.java} 

Im Codebeispiel \ref{lst:optinalsJava} ist zu erkennen, dass zur Compilezeit keine Aussage über den Zustand der Instanz-Variablen getroffen werden kann. Der Entwickler muss also aus dem logischen Kontext erkennen, welche Variablen eine null-Referenz enthalten könnten. Dies kann zu Laufzeitfehlern führen.

\lstinputlisting[caption=Swift Beispiel mit Optional Types, label=lst:optinalsSwift]{02_stateArt/codeExamples/OptionalType.swift}
In \ref{lst:optinalsSwift} sind die Instanzvariablen nun mit Optional Types modelliert. Nun 
kann zur Compilezeit zugesichert werden, welche Variablen eine null-Referenz enthalten können und welche Variablen sicher mit einem Wert belegt sind. Dies führt zu weniger Fehlern während der Laufzeit.

Auch APIs können mit \glqq Optionals\grqq  modelliert werden. So darf ein Parameter nicht null sein, wenn er nicht als Optional definiert wurde. Dies macht eine API strikter und führt ebenfalls zu weniger Laufzeitfehlern, da Fehler schon zur Compilezeit erkannt werden.

\section{Wearables}
Übersetzt man Wearables ins Deutsche, bedeutet es \glqq Tragbares\grqq  oder \glqq Anziehbares\grqq  im Sinne von einem Kleidungsstück tragen. Wenn man nun den Begriff Wearables mit Kleidung assoziiert, liegt man nicht falsch. Anstatt Computer auf dem Schreibtisch oder in der Hosentasche zu haben, trägt man sie am Körper wie Kleidungs- oder Schmuckstücke \cite{Dvorak:2008aa}. Die Grenzen zwischen Kleidung und Computer verschmelzen und es ist manchmal nicht klar, was man schon als Wearable bezeichnen kann oder auch nicht.

Wearables sind meist mit Sensoren ausgestattet, die Daten aus der Körperregion sammeln, an der sie getragen werden \cite{4711366}. Diese Daten zeigen immer nur einen Teilausschnitt. Durch das Tragen von mehreren Geräten am Körper verteilt, können mehr Daten gesammelt werden. So ist es möglich, einen noch genaueren Überblick über den Zustand des Körpers zu erhalten\cite{4711366}.

Erst mit der Miniaturisierung der Computertechnik und Sensoren war es möglich, Computer mit integriertem Akku in Größe einer Streichholzschachtel herzustellen. Während Swatch 1995/1996 eine Uhr vorstellte, die als Skipass funktionierte, stellte Apple 18 Jahre später die Apple Watch vor, die über einen Mikroprozessor, WLAN und eine Vielzahl an Sensoren verfügt (mehr zur Apple Watch in \ref{ch:apple-watch})

Doch nicht nur Uhren zählen zu den Wearables. Google präsentierte mit der Google Glass eine Datenbrille, die über eine Kamera, ein Heads \glqq Prismatic head-mount\glqq  Display, Sprachsteuerung sowie Internetverbindung und weitere Sensoren verfügt \cite{Muensterer2014281}. Auch ein Ring, der mit Hilfe eines Sensors den Puls misst, existiert als Prototyp \cite{4711366}.

Während im Smartphone-Markt noch ein Focus auf Leistung und Funktion der Geräte gelegte wurde, darf man beim Wearables-Markt den Faktor der Ästhetik nicht vergessen. Hier wird ein Bereich betreten, der starke Einflüsse von Mode aufzeigt. Anwender eines Gerätes achten also nicht mehr nur auf die Funktion, sondern auch auf Form, Farbe und Lifestyle den das Produkt verkörpert. Apple bietet unter dem Namen \glqq Watch Edition \grqq eine Apple Watch aus echtem Gold an, deren Preis über 10.000 Euro beträgt und sich an den Luxusmarkt richtet. Auch TAGHeuer, eine Firma, die sich auf luxuriöse und modische Uhren spezialisiert hat, betritt nun auch den Markt der Wearables \cite{TAGHeuer:20015aa}.

\chapter{Analyse und Entwurf}
\label{ch:probs-obs-procs}
 	%%
% ----------------------------------------------------------------------------
% "THE BEER-WARE LICENSE" (Revision 42):
% <sebastian.rauh@hs-heilbronn.de/michael.bauer@hs-heilbronn.de> wrote this 
% file. As long as you retain this notice you can do whatever you want with 
% this stuff. If we meet some day, and you think this stuff is worth it, you 
% can buy us a beer in return. 
% Michael Lukas Bauer, Sebastian Felix Rauh
% ----------------------------------------------------------------------------
%%
In diesem Kapitel Ergebnisse der Analyse und des Entwurfes erläuteret. große Teile der Anforderungsanalyse stammen aus dem PITA Praktikum.
\section{Entwicklungsmethodik}
Da das Projekt nur durch eine Person durchgeführt wurde, kann man nicht von einem definierbaren Methodik sprechen. Es handelte sich um ein iteratives Vorgehen. Funktionen, die bei der Evaluation erarbeitet wurden, sind direkt in eine neue Version der Anforderungen integriert worden. Der Quellcode wurde mit \gls{git} versioniert verwaltet
\section{Anforderungsanalyse}
Die Anforderungen wurden von eine Gerspräch mit Monika Pobiruchin am Anfang des Pita-Praktikum erhoben. Kernaussage dieser Anforderungen, dass das System von alten Menschen genutzt werden kann, um Medikamente regelmäßiger einzunehmen. Dies beinhaltet eine geringe Auseinandersetzung mit der Technik des Systems. Die Uhr soll möglichst autonom sein und nicht zwingen an ein Smartphone gekoppelt sein.

\section{Usecases} 
Von dem Gespräch mit Monika Pobiruchin wurde eine Persona für die Zielgruppe abgeleitet. Diese Persona ist im Anhang zu finden.
Mit Hilfe der Persona, wurden die Kern-Usacases abgeleitet. Die folgenden Usecase 1 bis 4 wurden im Praktikum erarbeitet und wurden von dort übernommen. 
\begin{usecase}
\addtitle{Usecase 1}{Der Patient wird an ein Medikament erinnert}
\addfield{Primärer Akteur}{Patient}
\addfield{Beschreibung}{Ein Erinnerungs Pop-Up erscheint auf dem Display und es wird ein Signal/eine Vibration ausgelöst. Das Pop-Up zeigt ein Abbild des Medikaments, dessen Namen und die Uhrzeit, zu der es eingenommen werden soll.}
\addfield{Vorbedingung}{Es wurde ein Medikationsplan aus der DB auf die Uhr geladen.}
\additemizedfield{Ablauf}{
\item Das Pop-Up mit der Erinnerung erscheint auf dem Smartwatch-Display. Die Erinnerung beinhaltet Informationen zur Uhrzeit, Menge und Art der Medikation
\item Der Patient bestätigt, dass er das Medikament genommen hat
\item Gerät bestätigt visuell, dass der Patient das Medikament als genommen markiert hat
}
\additemizedfield{Alternativablauf}{
\item Das Pop-Up mit der Erinnerung erscheint auf dem Smartwatch-Display
\item Der Patient wählt Option zum Verschieben der Medikation aus
\item Auf einem zusätzlichen Dialog kann er aus einer Auswahl eine Zeitdauer wählen, um die die Medikation verschoben wird}
\additemizedfield{Alternativablauf 2}{
\item Das Pop-Up mit der Erinnerung erscheint auf dem Smartwatch-Display
\item Der Patient reagiert nicht auf die Erinnerung
\item Die Erinnerung wird alle x Minuten wiederholt, solange der Patient nicht reagiert.
\item Das Fehlen einer Reaktion des Patientens innerhalb einer Zeit von x Minuten wird vermerkt}
\addfield{Ergebnis}{Der Patient hat die Einnahme des Medikaments bestätigt und dieses auch eingenommen}
\addfield{Alternativergebnis 1}{Der Patient hat die Erinnerung an die Medikamenteneinnahme verschoben}
\addfield{Alternativergebnis 2}{Der Patient hat die Erinnerung an das Medikament ausgeschaltet}
\end{usecase}

\begin{usecase}
\addtitle{Usecase 2}{Der Patient wird an mehrere Medikamente erinnert}
\addfield{Primärer Akteur}{Patient}
\addfield{Beschreibung}{Ein Erinnerungs Pop-Up erscheint auf dem Display und es wird ein Signal/eine Vibration ausgelöst. Das Pop-Up zeigt eine Liste der Medikamente, die eingenommen werden müssen.}
\addfield{Vorbedingung}{Es wurde ein Medikationsplan aus der DB auf die Uhr geladen.}
\additemizedfield{Ablauf}{
\item Das Pop-Up mit der Erinnerung erscheint auf dem Smartwatch-Display. Die Erinnerung beinhaltet Informationen zur Uhrzeit, Menge und Art der Medikationen
\item Der Patient bestätigt, dass er die Medikamente alle genommen hat
\item Gerät bestätigt visuell, dass der Patient die Medikamente als genommen markiert hat
}
\additemizedfield{Alternativablauf}{
\item Das Pop-Up mit der Erinnerung erscheint auf dem Smartwatch-Display
\item Der Patient wählt Option zum Verschieben der Medikationen aus
\item Auf einem zusätzlichem Dialog kann er aus einer Auswahl eine Zeitdauer wählen, um das die Medikationen verschoben werden}
\additemizedfield{Alternativablauf 2}{
\item Das Pop-Up mit der Erinnerung erscheint auf dem Smartwatch-Display
\item Der Patient wählt ein Medikament von der Liste aus
\item Patient befindet sich nun im Usecase 1}
\additemizedfield{Alternativablauf 3}{
\item Das Pop-Up mit der Erinnerung erscheint auf dem Smartwatch-Display
\item Der Patient reagiert nicht auf die Erinnerung
\item Die Erinnerung wird alle x Minuten wiederholt, solange der Patient nicht reagiert.
\item Das Fehlen einer Reaktion des Patientens innerhalb einer Zeit von x Minuten wird vermerkt}
\addfield{Ergebnis}{Der Patient hat die Einnahme der Medikamente bestätigt und diese auch eingenommen}
\addfield{Alternativergebnis 1}{Der Patient hat die Erinnerung an die Medikamenteneinnahme verschoben}
\addfield{Alternativergebnis 2}{Der Patient hat die Erinnerung an das Medikament ausgeschaltet}
\addfield{Alternativergebnis 3}{Der Patient hat bestimmte Medikamente ausgewählt und mit dem weiterführenden Usecase 1 bearbeitet}
\end{usecase}

\begin{usecase}
\addtitle{Usecase 3}{Einzelne Einnahmebestätigung zurücknehmen}
\addfield{Primärer Akteur}{Patient}
\addfield{Vorbedingung}{Es wurde ein Medikationsplan aus der DB auf die Uhr geladen. Eine Medikation wurde als genommen markiert}
\additemizedfield{Ablauf}{
\item Das System zeigt das genommene Medikament an
\item Der Benutzer drückt auf den Button mit der Aufschrift “Rücknahme”
\item Das System wechselt zur Darstellung eines einzelnen Medikaments, beschrieben im UseCase 1
}
\addfield{Ergebnis}{Die Einnahmebestätigung ist zurückgenommen. Die Erinnerung ist erneut zu bestätigen oder zu verschieben.}
\end{usecase}

\begin{usecase}
\addtitle{Usecase 4}{Mehrere Einnahmebestätigungen zurücknehmen}
\addfield{Primärer Akteur}{Patient}
\addfield{Vorbedingung}{Es wurde ein Medikationsplan aus der DB auf die Uhr geladen. Mehrere Medikationen, welche zur gleichen Zeit genommen wurden, wurde als genommen markiert}
\additemizedfield{Ablauf}{
\item Das System zeigt die genommenen Medikamente an
\item Der Benutzer drückt auf den Button mit der Aufschrift “Rücknahme”
\item Das System wechselt zur Darstellung mehrerer Medikamente, beschrieben im UseCase 2
}
\addfield{Ergebnis}{Die Einnahmebestätigung ist zurückgenommen. Die Erinnerung ist erneut zu bestätigen oder zu verschieben.}
\end{usecase}


Diese vier Usecases sind auch Grundlage für dies Arbeit. Wie die Usecases umgesetzt sind wird in Kapitel \ref{ch:realisation} beschrieben. In Kapitel \ref{ch:summ-eva-outl} finden sich überarbeitet Usecases, die Verbesserungen enthalten, welche aus der Evaluation mit der Zielgruppe hervorgehen.

\section{Apple Watch}
\label{ch:apple-watch}
Eine Andorderung, welche sich aus dem Kontext dieser Arbeit entnehmen lässt, ist die Nutzung der Apple Watch als Zielplatform. Die Apple Watch wurde im September 2014 vorgestellt und staret im April 2015 mit dem Verkauf.
\subsection{Hardware}
Die Apple Watch existiert in zwei Versionen. Einen Uhr mit 38mm (272x340) und eine mit 42mm (312x390) großem Display. Die bietet einen 8GB großen internen Speicher. Mit einer Akkulaufzeit von 18h unter durchschnittlicher Nutzung, hält die Uhr einen Tag durch \cite{Riches:2015aa}. 
\todo{Abhänigkeit zum iPhone}
\subsection{Software}
Mit erscheinen der Uhr wurde auch das Betriebsystem in Version 1 ausgeliefert und dazu das \gls{sdk} Names WatchKit. Dies erlaubte es Entwicklern Anwendungen zu Entwickeln, welche auf dem verbunden iPhone ausgeführt wurden. Diese führte zu schlechte Performance der Anwendungen und zu vollen Abhängigkeit zum iPhone.

Im Juni veröffentlichte Apple die erste Vorabversion von watchOS 2.0, welches später im September 2015 für Endnutzer bereit gestellt wurde. watchOS biete mehr Unabhängigkeit für Anwendungen. Die Andwendugen laufen direkt auf der Uhr. Für Anwendugsentwickler gibt es drei Informationen auf der Uhr dazustellen. Es Handelt sich um native Anwendugen (Apps), Glances, und Actionable Notifications \cite{Apple:2015devAw}.

Native Anwendungen sind fest installierte auf der Uhr. Sie können unabhängig auf der Uhr gestartet werden. In der Watch-App lassen sich komplexere Anwendungen realisieren, da der Nutzer durch viele Möglichkeiten der Interaktion Eingaben unter Interaktionen(\ref{ch:eingabe_interface} tätigen kann. Installiert werden die Apps vom iPhone aus. Eine Watch-App benötigt immer eine iPhone App, die jeweils auf dem iPhone installiert ist.

Ein Glance ist ein Interface, auf dem die wichtigesten Informationen einer App übersichtlich dargestellt werden. Der Nutzer soll mit einem Blick die Informationen erkennen. Es ist keine Interaktion mit einem Glance möglich. Tippt der der Nutzer auf einen Glance öffnet sich die Zugehörige App. Glances sind optional zu einer App zu entwickeln.




\subsection{Schniststellen}
Bluetooth 4.0 und Wi-Fi 802.11b/g sind die Netzwerkschnistellen der Apple Watch. Dazu kommt noch ein NFC Chip, der jedoch nicht über eine API nutzbar ist und vorerst nur für Apple Pay, dem Apple eigenen Bezahldienst, vorgesehen ist\cite{RITCHIE:2015aa}. 
\subsection{Sensoren}
Die Apple Watch besitzt einen Beschleunigungssensor und Gyroskop welche genaue Bewegungsdaten liefern. Ebenso ist ein optischer Herzschagsensor verbaut, der an der Unterseite der Uhr auf der Haut anliegt. Ein Mikrofon, welches für Spracheingabe genutzt werden kann ist auch vorhanden.
\subsection{Eingabe Interfaces}
\label{ch:eingabe_interface}
Neben eines normalen Touchescreen führte Apple in der Uhr auch eine Eingabeart Names Force Touch ein. Diese Technologie erlaubt es der Uhr zu erkennen, wie fest der Nutzer auf das Display drückt. Dies ermöglicht eine Neue Art der Eingabe. Besonders bei einer kleinen Eingabefläche, wie die der Apple Watch, ist eine neue Interaktionsdimesion interessant, da es eine differenziertere Interaktion erlaubt. Leider ist eine mögliche Interaktion mit Force Touch optisch nicht zu erkennen, was eine klare Nutzerführung schwer macht.

 Von Analogen Uhren ist die Krone, also das Rad an der Seite einer Uhr, an dem die Uhr eingestellt oder aufgezogen werden kann bekannt. An der Apple Watch ist die Krone auch zu finden. Apple nennt sie "Digital Crown", also digitale Krone. Sie befindet sich ebenfalls an der Seite der Uhr. Die Krone ist drehbar und lässt sich ebenfalls als Druckknopf nutzen. Sie dient zum scrollen von Inhalten, sowie präzisen Auswählen von Elementen aus einer Auswahlliste. Durch nutzen der Krone wird der Bildschirm nicht durch einen Finger verdeckt, was bei einem kleinen Display von Vorteil ist.

\subsection{Armband}
Apple biete 6 verschiede Armbänder für die Apple Watch an (Stand Nov. 2015). Zusätzlich ist es möglich, Armbänder von Drittherstellern zu erwerben. 

Das in der günstigsten Version mitgelieferter Version (Sportarmband) verfügt über einen sehr komplizierten Verschlussmechanismus und ist deswegen weniger für Menschen geeignet, die über schwache sensomotorische Fähigkeiten verfügen. Es gibt auch Armbänder, die einen magnetischen Verschluss bieten, diese biete eine einfachen Handhabung, sind jedoch 3 mal so teuer, wie das Sportarmband.

An der Verbindung zwischen Armband und Uhr ist eine nicht weiter spezifizierter Wartungsport verbaut. Dieser Anschluss könnte in Zukunft Armbänder ermöglichen, welche Informationen aus dem Armband an die Uhr weiterleiten.



 	
\chapter{Umsetzung}
\label{ch:realisation}
 	%%
% ----------------------------------------------------------------------------
% "THE BEER-WARE LICENSE" (Revision 42):
% <sebastian.rauh@hs-heilbronn.de/michael.bauer@hs-heilbronn.de> wrote this 
% file. As long as you retain this notice you can do whatever you want with 
% this stuff. If we meet some day, and you think this stuff is worth it, you 
% can buy us a beer in return. 
% Michael Lukas Bauer, Sebastian Felix Rauh
% ----------------------------------------------------------------------------
%%

Die in Kapitel 3 aufgeführten Analyse Ergebnisse wurden für die erste Interation des Prototyps umgesetzt. Es handelt sich hierbei um ein funktionsfähigen Prototypen, der nativ auf der Apple Watch ausführbar ist. Im folgenden werden Schritte der Umsetzung genauer beschrieben. Hierbei wird genauer auf die Benutzeroberflächenerstellung, sowie auch die Verbindung zischen Uhr und iPhone eingegangen.

\section{Benutzeroberfläche}
Xcode bietet für visuelle Erstellung von Benutzeroberflächen ein eigenen integrierten Editor namens InterfaceBuilder bereit. Hiermit können graphische Elemente per DragAndDrop zu einer Benutzeroberfläche zusammengestellt werden \ref{fig:xcode-interface-elements}. Ebenfalls per DragAndDrop werden diese Inerface-Elemete mit dem Quellcode verbunden\ref{fig:xcode-interface-code-connect}.
\begin{figure}
	\caption{Interface Elemente zu Erstellen von Benutzeroberflächen}
	\label{fig:xcode-interface-elements}
	\centering
		\includegraphics[width=0.9\textwidth]{04_realisation/screenshots/xcode-interface-elements}
\end{figure}

\begin{figure}
	\caption{Interface Elemente mit Quellcode verknüfen}
	\label{fig:xcode-interface-code-connect}
	\centering
	\includegraphics[width=0.9\textwidth]{04_realisation/screenshots/xcode-interface-code-connect}
\end{figure}

\section{WatchConnectivity}
Wichtig für die Kommunikation zwischen Uhr und iPhone ist das WatchConnectivity Framework. Hierbei ist ist im Listing xx zu sehen, wie genau eine Verbindung aufgebaut werden kann. Wichtig ist, das diese Verbindung zum richtigen Zeitpunkt im Application-Lifecycle aufgebaut wird, da es sonst zu Datenverlusten kommen kann.
\lstinputlisting[caption=Beispiel zu Type Inference in Swift label=lst:watchConnectifity]{04_realisation/code/WatchExecutionTimeService.swift}


\section{Anwendung}


 	
\chapter{Evaluation}
\label{ch:summ-eva-outl}
 	%%
% ----------------------------------------------------------------------------
% "THE BEER-WARE LICENSE" (Revision 42):
% <sebastian.rauh@hs-heilbronn.de/michael.bauer@hs-heilbronn.de> wrote this 
% file. As long as you retain this notice you can do whatever you want with 
% this stuff. If we meet some day, and you think this stuff is worth it, you 
% can buy us a beer in return. 
% Michael Lukas Bauer, Sebastian Felix Rauh
% ----------------------------------------------------------------------------
%%
In Kapitel 5 werden die erarbeiteten Ergebnisse mit Probanden der Zielgruppe evaluiert. Hier werden Schwächen und Stärken des erstellten Konzepts sichtbar.

\section{Evaluation}
Die Evaluierung des Prototyps soll ein quantifiziertes Ergebnis liefern, anhand dessen Schwachstellen aufgezeigt werden sollen. Diese Schwachstellen werden mit der nächsten Iteration der Software ausgebessert.

\subsection{Zielgruppe der Evaluation}
Die Zielgruppe sollten Menschen sein, deren Alltag von Medikamenteneinnahmen geprägt ist. Es sollte sich um Menschen mit wenig oder keinen Vorkenntnissen mit touchscreen-basierten Geräten handeln.  Patienten fortgeschrittenen Alters, die sich stationär im Krankenhaus aufhalten, eignen sich gut. Diese nehmen in der Regel Medikamente und haben viel freie Zeit während des Aufenthaltes im Krankenhaus für eine Befragung.

\subsection{Vorgehen}
\label{ch:vorgehen}
Im ersten Schritt wird dem Probanden die Apple Watch gezeigt und die dahinterliegende Technologie erklärt, um Neugier bei dem Patienten zu wecken. Nun können die Patienten die Uhr selbstständig anlegen. 
Im zweiten Schritt wird eine Notification, die eine Vibration auslöst, gestartet. Die Patienten müssen die Vibration spüren und dann auf die Uhr schauen. Dort sollen sie den Bildschirminhalt wiedergeben. Haben sie diese Aufgabe erfolgreich gelöst, so müssen sie die Benachrichtigung bestätigen, also auf den Button mit der Aufschrift \glqq Genommen \grqq (siehe \ref{fig:watch-app-notification}) tippen.

In Schritt Drei müssen die Patienten den App-Bildschirm durch Drücken auf die Digital Crown aufrufen, um dort die Mediwatch App zu öffnen. Wenn die App geöffnet ist, soll ein Medikament ausgewählt werden und dessen Einnahmezeitpunkt um eine definierte Zeitdauer verschoben werden. Ist der Einnahmezeitpunkt verschoben, so ist die Aktion beendet.

\subsection{Auswertung}
Für die Evaluation werden zwei Fragebögen genutzt. Einmal AttrakDiff\cite{UserInDe:Attrakdiff}, welcher die subjektive  Wahrnehmung der Bedienbarkeit und das Aussehen des Prototyps erfragt. Es handelt sich um einen standardisierten Fragebogen und eine standardisierte Auswertungsmethode. Da dieser Fragebogen keine Antworten über Funktionen des Prototyps gibt, ist es nötig, einen zweiten Fragebogen zu erstellen. Dieser erfragt die Situation, also den Kontext in dem sich der Anwender befindet und die daraus folgenden Ansprüche an den Prototyp. So sollen fehlende Funktionen oder Fehler in der Analyse aufdeckt werden. Die Fragen sind im Anhang zu finden.
Die Erfassung der Fragen wird mit Limesurvey\cite{Limesurvey} in Version 2.05 durchgeführt. Da sich Limesurvey selbst hosten lässt, bleiben die erfragten Daten auf einem sicheren Hochschulserver und werden nicht bei Drittanbietern gespeichert. 

\subsection{Methodik - Thinking Aloud}
\label{ch:thinking}
Die Thinking Aloud nach Nielsen \cite{Nielsen:1993aa} beschreibt eine Methode zur Feedback-Gewinnung von Software. Probanden werden aufgefordert während der Nutzung der Software und deren Nutzerschnittstelle ihre Gedanken laut auszusprechen. Hierfür sind passende Probanden zu finden, die jeweils eine vorgegebene Aufgabe mit der Software ausführen (siehe \ref{ch:vorgehen}).

Die Vorteile diese Methode liegen auf der Hand. Sie ist sehr günstig, da keinerlei Geräte verwendet werden. Auch muss sie nicht in einem Labor durchgeführt werden. Es ist sogar von Vorteil die Anwendung im üblichen Benutzungskontext der Anwendug zu testen. Meist reicht schon eine kleine Anzahl an Probanden um ein deutliches Feedback zu bekommen. Die Methodik erweißt sich als sehr robust, da kaum Fehler bei der Durchführung gemacht werden können. Die Methode kann zu jedem Zeitpunkt im Entwicklungsprozess eingesetzt werden. So eigent sie sich sehr gut für agile Prozesse bei denen Prokektteile flexibel getestetet werden sollen. Durch die niedrige Barriere zur Durchführung können auch Softwareentwickler oder Manager diesen Prozess durchführen und erhalten dadurch sehr schnell Feedback. Da diese Feedback sehr persönlich ist, wird es stärker von den Testern aufgenommen als Ergebnisse die reine Zahlen auf Papier.

Da die die meisten Menschen es nicht gewohnt sind Selbstgespräche zu führen, sollte darauf geachtet werden den Probanden immer im Redefluss zu halten. Auch kann es für den Durchführenden schwierig sein zwischen unwichtigen Aussagen und wichtigen Aussagen desProbanden zu unterscheiden. Der Durchführende muss darauf achten, dass er mit den Anweisung den Probanden nicht beeinflusst und so das Ergebnis verfälscht.

\section{Durchführung der Evaluation}
Die Befragung wurde im Kreiskrankenhaus Mosbach unter Aufsicht von Frau Flohr durchgeführt. Die Patienten befanden sich stationär auf der geriatrischen Rehabilitation. Es haben insgesamt 11 Patienten  an der Befragung teilgenommen. Davon waren 7 weiblich und 4 männlich. Alle waren im Alter zwischen 70 und 85. 

\subsection{Befragung der Patienten}

Die Befragung in der Zielgruppe verlief sehr schwierig und nicht wie geplant.
Der erste Schritt der Befragung klappte bei fast allen Probanden sehr gut. So erkannten sie die Notification mit der einhergehenden Vibration. Der Bildschirminhalt wurde von fast allen Patienten als verständlich geschildert. Das darauf folgende Bestätigen einer Notification verlief zu großen Teilen schlecht. Die Patienten konnten die Buttons zum Bestätigen nicht treffen, da bei ihnen die Feinmotorik eingeschränkt war.
Das Öffnen einer App ist aufgrund der reduzierten Feinmotorik ebenfalls nicht möglich. Die App-Icons sind zu klein und werden nicht getroffen. Das Öffnen der Anwendung wurde bei allen Patienten nicht erreicht.
\subsection{Auswertung - Ergebnis}
Der Fragebogen AttrakDiff ist in dieser Zielgruppe nicht praktikabel. Eine sehr genaue Unterscheidung zwischen den Begrifflichkeiten, die der  AttrakDiff abfragt, ist für die Zielgruppe nicht möglich. Dies fällt auf, wenn man mit den Probanden spricht. Sie schweifen oft ab und man ist nicht in der Lage, ihre Aussagen zu erfassen. Direkte Fragen, die den Alltag der Probanden betreffen, werden zuverlässig beantwortet. Schwierigkeiten, die die Patienten mit der Medikamenteneinnahme haben,  werden nicht zugegeben bzw. heruntergespielt.
Die Befragungen wurden deswegen auf eine Thinking Alound Methode (\ref{ch:thinking}) umgestellt. Leider sind die Ergebnisse so nicht mehr quantifizierbar, jedoch sind aus den Aussagen der Patienten und den Beobachtungen gute Schlussfolgerungen möglich.
Auch der zweite Fragebogen zu den Funktionen und deren Nutzungskontext wurde nicht verwendet. Hier wurde versucht, während des Gesprächs  mehr Einblick in das Leben der Probanden zu erhalten. Probanden erzählen gerne über ihr Leben und suchen eher das persönliche Gespräch. Daraus geht hervor, dass die Patienten mit großer Übereinstimmung keinen Internetanschluss besitzen. Dies schließt entfernte Aktualisierungen des Medikationsplans und Einnahmebestätigungen des Patienten für den Arzt aus. Missgefallen äußern die weiblichen Patientinnen über die Größe und das Aussehen der Uhr. Eine Uhr muss dem Geschmack der Patientin entsprechen. Hierbei spielt die Größe wie auch das Aussehen der Uhr eine Rolle. Bei Beschreibung von anderen modischen Farben der Uhr oder der Armbänder zeigen sie sich interessiert. Ein Großteil der Patienten sieht jedoch ein, dass sie aufgrund ihrer Sehschwäche die große Uhr (42mm) benötigen. Die große Uhr wird im Gegenzug als zu groß beschrieben. Die meisten Patienten haben keine Probleme die Schrift zu lesen. Meist nehmen sie ihre Brille zur Hilfe, die im Krankenhaus natürlich immer griffbereit ist. Nur bei einem kleinen Teil kommt es vor, dass sie den Text auf dem Display auch mit Brille nicht lesen können. Das größte Problem ist jedoch die schon erwähnte Einschränkung der Feinmotorik. So können die Probanden die Uhr nicht wirklich aktiv bedienen, sondern reagieren nur auf die Vibration am Handgelenk. Dies führt zu einer Hilflosigkeit gegenüber der Technik der Uhr. Eine genaue Auflistung der Probleme findet sich in \ref{ch:problems}
\begin{table}[]
\centering
\caption{Probleme und mögliche Lösungen des Smartwatch Prototyps}
\label{ch:problems}
\begin{tabular}{p{4cm} p{5cm} p{5cm}|l|l|l|}
\hline
 Problem  &Auswirkungen  &mögliche Lösung  \\ \hline
 \textbf{fehlender Internetanschluss}  &keine Aktualisierungen des Medikamentenplans und keine Benachrichtigungen für den Arzt &Uhr oder Smartphone mit Mobilfunkverbindung ausstatten  \\
 \textbf{reduzierte Feinmotorik der Probanden} &Buttons werden nicht getroffen oder falsche Aktionen werden ausgelöst  &große haptische Knöpfe in die Hardware integrieren, die gut drückbar sind  \\
 \textbf{Vibration zu leicht}&Vibration wird nicht gespürt und Medikament wird vergessen  &Stärkere Vibration, lauten Ton dazu abspielen, starkes optisches Feedback (grelles Blinken)   \\
 \textbf{Nur mit Brille lesbar} &Benachrichtigung wird erkannt, bis jedoch die Brille im Haushalt gefunden, ist die Medikation schon vergessen oder die Uhr hat die Benachrichtigung zurückgestellt  &Größere Schrift, die auch ohne Brille lesbar ist. Audiowiedergabe der Medikation  \\
 \textbf{Armband schwer anlegbar}&Uhr wird morgens nicht gerne angezogen,  bleibt liegen und Patient bekommt keine Benachrichtigungen  &Verzicht auf Standard Sport-Band und dafür Armbänder, die leicht anzuziehen sind, verwenden. Viele Patienten tragen dehnbare Bänder ohne Verschluss \\
 \textbf{Uhr ist unästhetisch}&Patient trägt die Uhr nicht und wird so nicht an die Medikamente erinnert  &Keine Standard Uhren kaufen, sondern den Patienten die Farbe und Form der Uhr und die Art des Armbandes aussuchen lassen  \\ \hline
\end{tabular}
\end{table}
\subsection{Auswertung des Fragebogens zum Nutzungskontext}
Die Frage des Fragebogen zum Nutzungskontext wurde im Laufe des Gesprächs oder im Nachgang des Thinking Alouds geklärt. Die Aussagen waren jedoch oft nicht erkennbar, also handelt es sich eher um eine Einschätzung der Antworten. Die Antworten sind in \ref{fig:result} graphisch dargestellt.

\begin{figure}
\begin{tikzpicture}
[
    pie chart,
    slice type={comet}{blu},
    slice type={legno}{rosso},
    pie values/.style={font={\small}},
    scale=2
]


	\pie[shift={(0cm, 0cm)}]{Tragen eine Uhr im Alltag}{73/comet, 27/legno}
    \pie[shift={(4.3cm, 0cm)}]{Besitzen einen Computer undInternetanschluss}{9/comet, 91/legno}
    \pie[shift={(0cm, -2.6cm)}]{Findet die Idee gut }{81/comet, 19/legno}
    \pie[shift={(4.3cm, -2.6cm)}]{Haben Probleme Medikamente rechtzeitig zu nehmen}{36/comet, 64/legno}
    \pie[shift={(0cm, -5.2cm)}]{Wissen welche Medikamente sie nehmen}{73/comet, 27/legno}
    \pie[shift={(4.3cm, -5.2cm)}]{Erkennen Medikamente an Form u. Farbe}{45/comet, 65/legno}
    \pie[shift={(0cm, -7.8cm)}]{Besitzen ein mobiles Telefon}{27/comet, 73/legno}
    \pie[shift={(4.3cm, -7.8cm)}]{Würden sie Anwendung + Uhr nutzen}{36/comet, 64/legno}
    
    

	\centering
    \legend[shift={(2.2cm,-8.6cm)}]{{Ja}/comet, {Nein}/legno}

\end{tikzpicture}
\caption{Ergebnis des Fragebogens für den Nutzungskontext}
\label{fig:result}
\end{figure}
\section{Probleme bei der Evaluation}
Die Zielgruppe mit dem Alter zwischen 70 und 85 erwies sich als schwierig. Eine bessere Vorbereitung auf diese Gruppe von Menschen wäre von Vorteil gewesen. Die Fragen, die geprüft werden, sollten auf eine minimale Anzahl reduziert werden. Es ist eher wichtig einen persönliche Verbindung zum Befragten am Anfang des Gesprächs aufzubauen. Konfrontiert man sie gleich mit einer großen Anzahl an Fragen, verunsichert man sie. Auch sollte man hierbei geduldig sein und ein persönliche Themen aufgreifen. Die Befragten stehen anfangs einer Befragung eher skeptisch gegenüber. Diese Skepsis gilt es zu überwinden.
 	
\chapter{Zusammenfassung und Ausblick}
\label{ch:summ-eva-outl}
 	%%
% ----------------------------------------------------------------------------
% "THE BEER-WARE LICENSE" (Revision 42):
% <sebastian.rauh@hs-heilbronn.de/michael.bauer@hs-heilbronn.de> wrote this 
% file. As long as you retain this notice you can do whatever you want with 
% this stuff. If we meet some day, and you think this stuff is worth it, you 
% can buy us a beer in return. 
% Michael Lukas Bauer, Sebastian Felix Rauh
% ----------------------------------------------------------------------------
%%

\section{Zusammenfassung}
Ziel der Arbeit war es, die Akzeptanz eines Systems zur Erinnerung an Medikamenteneinnahme zu testen. Als Methode wurde eine Patientenbefragung verwendet. Patienten stehen dem Tragen einer Uhr am Handgelenk positiv gegenüber. Ob sie jetzt digital oder analog ist, ist nebensächlich. Aus diesen Gründen ist die Uhr ein ideales Gerät zur Erinnerung von Medikamenten. Die Apple Watch mit ihrer touchscreen-basierten Navigation stellt für die getestete Zielgruppe eine große Herausforderung dar. Reduzierte feinmotorische Fähigkeiten erschweren den Patienten die Interaktion mit der Uhr. Auch Vibration und Geräuschwiedergabe sind zu schwach und können häufig nicht wahrgenommen werden. Das ästhetische Aussehen der Uhr ist nicht zu vernachlässigen, da eine Uhr oft als eine Art Schmuck getragen wird. So trägt die Farbe, Form und Größe zur Akzeptanz der Uhr bei. Findet der Patient die Uhr unschön/unmodisch, so wird er sie nicht tragen und schlimmstenfalls seine Medikamente vergessen.

Da Smartwatches noch keine große Verbreitung haben, sind auch die Werkzeuge zur Umsetzung von Anwendungen begrenzt. Es gibt keine Werkzeuge zur Erstellung interaktiver Prototypen, die auf der Uhr ausführbar sind. Ein Prototyp kann ausschließlich unter Verwendung von Code erstellt werden. So vermehrt sich der Aufwand für schnelles iteratives Vorgehen. Das watchOS SDK von Apple bietet eine relativ übersichtliche Schnittstelle. Die Schnittstellen ermöglichen es komplexere Anwendungen zu erstellen. Diese sind jedoch in manchen Anwendungsgebieten noch limitiert. Da es sich um die erste Geräte-Generation handelt, muss abgewartet werden mit welchen Leistungsverbesserungen weitere Generationen nachgerüstet werden, um die Uhr zu einem wirklich nützlichen Werkzeug zu machen. Das Potential zu einem guten Nutzen zeigt die erste Generation auf, jedoch ist sie an manchen Stellen (Geschwindigkeit, lange Wartezeiten) noch nicht ausgereift.

Apples neue Programmiersprache Swift, die zur Implementierung genutzt wurde, bietet Konzepte moderner Programmiersprachen. Diese Konzepte unterstützen den Entwickler meist schon zur Compilezeit. Dies führt zu einer frühen Fehlererkennung. Durch diese Spracheigenschaften, wie Optionals (\ref{ch:optionals}), die Sicherheit in der Modellierung geben, oder Closures und First Class Functions (\ref{ch:closures}), die Konzepte funktionaler Programmierung bereitstellen, eignet sich Swift sehr gut als Lehrsprache. Dadurch, dass Swift unter einer Open Source Lizenz veröffentlicht wurde, kann man nun für mehrere Zielplattformen entwickeln.

\section{Ausblicke}
Mit den gewonnenen Ergebnissen kann dieses Projekt neu ausgerichtet werden. Zum Beispiel sollte die Zielgruppe über mehr Erfahrung mit digitalen Geräten verfügen. Der Einsatz der Uhr bei Kindern und Jugendlichen im Alter ab 12 Jahren ist denkbar. Diese könnten bei ihrer Therapie unterstützt und durch die moderne Technik, mit der sie aufgewachsen sind, motiviert werden Medikamente zeitgerecht einzunehmen.
Nachdem die Defizite mit der Bedienbarkeit durch die Evaluation entdeckt wurden, ergab die Recherche, dass die watchOS Plattform über Funktionen der Accessibility verfügt \cite{Apple:watchAccess} . Damit ist es möglich, Nutzungs-Erleichterungen für Menschen mit körperlichen Einschränkungen zu schaffen. So kann mit DynamicType, also eine dynamischen Schriftgröße in der App, die Lesbarkeit verbessert werden. Der Nutzer kann so seine eigene Schriftgröße wählen. Auch die Navigation zum Öffnen von Anwendungen wird mit Accessibility Einstellungen verbessert. Die Uhr als Hilfsmittel im Alltag zeigt gute Ansätze auf und ist in der Zukunft weiter zu verfolgen. Wichtig hierbei ist, dass die Uhr als solches bestehen bleibt und nicht durch einen großen unschönen Kasten am Handgelenk ersetzt wird. 




\twocolumn 
\bibliography{06_appendix/03_bibliography/bib}

\onecolumn
\appendix 	
\chapter{Appendix}
\label{ch:appendix}
 	%
% ----------------------------------------------------------------------------
% "THE BEER-WARE LICENSE" (Revision 42):
% <sebastian.rauh@hs-heilbronn.de/michael.bauer@hs-heilbronn.de> wrote this 
% file. As long as you retain this notice you can do whatever you want with 
% this stuff. If we meet some day, and you think this stuff is worth it, you 
% can buy us a beer in return. 
% Michael Lukas Bauer, Sebastian Felix Rauh
% ----------------------------------------------------------------------------


\section{Fragebogen}
\begin{enumerate}
\item Tragen Sie eine Uhr im Alltag
\begin{tasks}(4)
\task Ja
\task Nein
\end{tasks}
\item Wären Sie bereit eine Digitale Computer Uhr zu tragen
\begin{tasks}(4)
\task Ja
\task Nein
\end{tasks}
\item Finden Sie die Idee gut, von der Uhr an ihre Einnahme erinnert zu werden.
\begin{tasks}(4)
\task Ja
\task Nein
\end{tasks}
\item Finden Sie die Idee gut, von der Uhr an ihre genauen Medikamente erinnert zu werden?
\begin{tasks}(4)
\task Ja
\task Nein
\end{tasks}
\item Wie viele Medikamente nehmen Sie täglich
\begin{tasks}(4)
\task ein Medikament
\task mehr als drei
\task mehr als fünf
\task mehr als acht
\end{tasks}
\item Haben Sie Probleme, Ihre Medikamente rechtzeitig zu nehmen?
\begin{tasks}(4)
\task Ja
\task Nein
\end{tasks}
\item Wissen Sie welche Medikamente die nehmen?
\begin{tasks}(4)
\task Ja
\task Nein
\end{tasks}
\item Können Sie die Medikamente an Form und Farbe unterscheiden
\item Haben Sie einen Internetzugang zu Hause?
\begin{tasks}(4)
\task Ja
\task Nein
\end{tasks}
\item Benutzen Sie einen Computer?
\begin{tasks}(4)
\task Ja
\task Nein
\end{tasks}
\item Benutzen Sie ein mobiles Telefon?
\begin{tasks}(4)
\task Ja
\task Nein
\end{tasks}
\item Würden Sie die Erinnerung an die Medikamente mit der Uhr nutzen?
\begin{tasks}(4)
\task Ja
\task Nein
\end{tasks}
\item Welche Funktionen fehlen, die Sie als wichtig ersehen?
\item Gibt es Funktionen die Sie für unwichtig halten?
\end{enumerate}
%
%\section{List of Abbrevations}
%\label{sec:list-abbrevations}

%\section{Acknowlagement}
%\label{sec:acknowlagement}

%\section{Affirmation}
%\label{sec:affirmation} 
 	
\chapter*{Erklärung der Urheberschaft}

Ich erkläre hiermit an Eides statt, dass ich die vorliegende Arbeit
ohne Hilfe Dritter und ohne Benutzung anderer als der angegebenen
Hilfsmittel angefertigt habe; die aus fremden Quellen direkt oder
indirekt übernommenen Gedanken sind als solche kenntlich gemacht. Die
Arbeit wurde bisher in gleicher oder ähnlicher Form in keiner anderen
Prüfungsbehörde vorgelegt und auch noch nicht veröffentlicht.


\vspace{4cm}

\hspace{2cm} Ort, Datum \hfill Unterschrift \hspace{2cm}
\end{document}

