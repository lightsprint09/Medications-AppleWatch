\section{Motivation}
Durch steigende Anzahl an Medikamenten, die Patienten über den Tag nehmen müssen, kann dies zu einer großen mentalen Aufgabe für den Patienten werden. Durch Hilfsmittel wie Medikationspläne oder nach Zeit vorsortierte Medikamente, kann die Planung der Einnahme erleichtert werden. 

Zur Erleichterung der Patienten soll der Medikationsplan nun per Smartwatch einsehbar gemacht werden und somit dem Patienten eine Interaktion mit dem Plan ermöglichen. Daraus bieten sich auch Vorteile für den behandelden Arzt, der Einblick in die Einnahmegewohnheiten seines Patienten bekommt. Die initialen Ideen stammen vom \gls{pita} an der HS-Heilbronn. An der HS-Heilbronn wird eine Komponente für Ärzte entwickelt, die es ihnen ermöglicht, die Medikationen der Patienten zu pflegen, zu überwachen und auszuwerten.
\section{Zielstezung}
Die Ziele der Arbeit sind wie folgt.

Es wird ein interaktiver Prototyp entwickelt, welcher auf der Apple Watch ausgeführt werden kann. Dieser Prototyp soll dann mit geeigneten Probanden, die zur passenden Zielgruppe gehören, evaluiert werden.
Die Ergebnisse der Evaluierung sollen im Rahmen der technischen Möglichkeiten umgesetzt werden. Zum Schluss soll noch die Machbarkeit überprüft werden, die Anwendung an das Backend-System vom PITA-Praktikum anzuschließen.

\section{Aufbau der Arbeit}

In Kapitel 2 wird auf das Forschungsumfeld der Arbeit Bezug genommen. Weiter werden technologische Grundlagen beschrieben wie Apple`s Swift Programmiersprache und die Technologie von Wearables.
Im drittel Kapitel werden Anforderungen aufgeführt, die teilweise aus dem  PITA-Praktikum stammen. Weiter werden Anforderungen geschildert, die sich aus den Möglichkeiten der Apple Watch ergeben.
Abschnitt 4 betrachtet die Kernpunkte der Implementierung. Weiter wird hier der Prototyp beschrieben.
In Kapitel 5 wird die Planung und Durchführung der Evaluierung dargestellt. Die  Ergebnisse der Evaluierung an der Zielgruppe sind hier ebenfalls zu finden.
In der Diskussion in Kapitel 6 wird aufgezeigt, welche Probleme während der Arbeit entstanden sind. Es wird gezeigt, wie der Prototyp im Gesamtbild einzuordnen ist und es wird ein Ausblick gegeben, welche Ziele mit dem System weiter verfolgt werden können.

