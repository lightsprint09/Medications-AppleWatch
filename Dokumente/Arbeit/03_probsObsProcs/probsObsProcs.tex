%%
% ----------------------------------------------------------------------------
% "THE BEER-WARE LICENSE" (Revision 42):
% <sebastian.rauh@hs-heilbronn.de/michael.bauer@hs-heilbronn.de> wrote this 
% file. As long as you retain this notice you can do whatever you want with 
% this stuff. If we meet some day, and you think this stuff is worth it, you 
% can buy us a beer in return. 
% Michael Lukas Bauer, Sebastian Felix Rauh
% ----------------------------------------------------------------------------
%%
Im Kapitel drei werden Ergebnisse der Analyse und des Entwurfes dargestellt. Große Teile der Anforderungsanalyse stammen aus dem \gls{pita}. Die Hardware und Software der Apple Watch wird genauer beschrieben.
\section{Entwicklungsmethodik}
Da das Projekt nur durch eine Person durchgeführt wurde, kann man nicht von einer definierbaren Methodik sprechen. Es handelte sich um ein iteratives Vorgehen. Funktionen, die bei der Evaluation erarbeitet wurden, sind direkt in eine neue Version der Anforderungen integriert worden. Der Quellcode wurde mit \gls{git} versioniert verwaltet.
\section{Anforderungsanalyse}
Die Anforderungen wurden von einem Gespräch mit Monika Pobiruchin am Anfang vom \gls{pita} erhoben. Kernaussage dieses Gesprächs war, dass das System von alten Menschen genutzt werden soll, um Medikamente regelmäßiger einzunehmen. Dies beinhaltet eine geringe Auseinandersetzung mit der Technik des Systems. Die Uhr soll möglichst autonom sein und nicht zwingend an ein Smartphone gekoppelt sein.

\section{Usecases} 
Von dem Gespräch mit Monika Pobiruchin wurde eine Persona für die Zielgruppe erstellt.
Mit Hilfe der Persona wurden die Kern-Usecases abgeleitet. Die folgenden Usecases 1 bis 4 wurden im \gls{pita} erarbeitet und von dort übernommen. 
\begin{usecase}
\addtitle{Usecase 1}{Der Patient wird an ein Medikament erinnert}
\addfield{Primärer Akteur}{Patient}
\addfield{Beschreibung}{Ein Erinnerungs Pop-Up erscheint auf dem Display und es wird ein Signal/eine Vibration ausgelöst. Das Pop-Up zeigt ein Abbild des Medikaments, dessen Namen und die Uhrzeit, zu der es eingenommen werden soll.}
\addfield{Vorbedingung}{Es wurde ein Medikationsplan aus der DB auf die Uhr geladen.}
\additemizedfield{Ablauf}{
\item Das Pop-Up mit der Erinnerung erscheint auf dem Smartwatch-Display. Die Erinnerung beinhaltet Informationen zur Uhrzeit, Menge und Art der Medikation
\item Der Patient bestätigt, dass er das Medikament genommen hat
\item Gerät bestätigt visuell, dass der Patient das Medikament als genommen markiert hat
}
\additemizedfield{Alternativablauf}{
\item Das Pop-Up mit der Erinnerung erscheint auf dem Smartwatch-Display
\item Der Patient wählt Option zum Verschieben der Medikation aus
\item Auf einem zusätzlichen Dialog kann er aus einer Auswahl eine Zeitdauer wählen, um die die Medikation verschoben wird}
\additemizedfield{Alternativablauf 2}{
\item Das Pop-Up mit der Erinnerung erscheint auf dem Smartwatch-Display
\item Der Patient reagiert nicht auf die Erinnerung
\item Die Erinnerung wird alle x Minuten wiederholt, solange der Patient nicht reagiert.
\item Das Fehlen einer Reaktion des Patientens innerhalb einer Zeit von x Minuten wird vermerkt}
\addfield{Ergebnis}{Der Patient hat die Einnahme des Medikaments bestätigt und dieses auch eingenommen}
\addfield{Alternativergebnis 1}{Der Patient hat die Erinnerung an die Medikamenteneinnahme verschoben}
\addfield{Alternativergebnis 2}{Der Patient hat die Erinnerung an das Medikament ausgeschaltet}
\end{usecase}

\begin{usecase}
\addtitle{Usecase 2}{Der Patient wird an mehrere Medikamente erinnert}
\addfield{Primärer Akteur}{Patient}
\addfield{Beschreibung}{Ein Erinnerungs Pop-Up erscheint auf dem Display und es wird ein Signal/eine Vibration ausgelöst. Das Pop-Up zeigt eine Liste der Medikamente, die eingenommen werden müssen.}
\addfield{Vorbedingung}{Es wurde ein Medikationsplan aus der DB auf die Uhr geladen.}
\additemizedfield{Ablauf}{
\item Das Pop-Up mit der Erinnerung erscheint auf dem Smartwatch-Display. Die Erinnerung beinhaltet Informationen zur Uhrzeit, Menge und Art der Medikationen
\item Der Patient bestätigt, dass er die Medikamente alle genommen hat
\item Gerät bestätigt visuell, dass der Patient die Medikamente als genommen markiert hat
}
\additemizedfield{Alternativablauf}{
\item Das Pop-Up mit der Erinnerung erscheint auf dem Smartwatch-Display
\item Der Patient wählt Option zum Verschieben der Medikationen aus
\item Auf einem zusätzlichem Dialog kann er aus einer Auswahl eine Zeitdauer wählen, um das die Medikationen verschoben werden}
\additemizedfield{Alternativablauf 2}{
\item Das Pop-Up mit der Erinnerung erscheint auf dem Smartwatch-Display
\item Der Patient wählt ein Medikament von der Liste aus
\item Patient befindet sich nun im Usecase 1}
\additemizedfield{Alternativablauf 3}{
\item Das Pop-Up mit der Erinnerung erscheint auf dem Smartwatch-Display
\item Der Patient reagiert nicht auf die Erinnerung
\item Die Erinnerung wird alle x Minuten wiederholt, solange der Patient nicht reagiert.
\item Das Fehlen einer Reaktion des Patientens innerhalb einer Zeit von x Minuten wird vermerkt}
\addfield{Ergebnis}{Der Patient hat die Einnahme der Medikamente bestätigt und diese auch eingenommen}
\addfield{Alternativergebnis 1}{Der Patient hat die Erinnerung an die Medikamenteneinnahme verschoben}
\addfield{Alternativergebnis 2}{Der Patient hat die Erinnerung an das Medikament ausgeschaltet}
\addfield{Alternativergebnis 3}{Der Patient hat bestimmte Medikamente ausgewählt und mit dem weiterführenden Usecase 1 bearbeitet}
\end{usecase}

\begin{usecase}
\addtitle{Usecase 3}{Einzelne Einnahmebestätigung zurücknehmen}
\addfield{Primärer Akteur}{Patient}
\addfield{Vorbedingung}{Es wurde ein Medikationsplan aus der DB auf die Uhr geladen. Eine Medikation wurde als genommen markiert}
\additemizedfield{Ablauf}{
\item Das System zeigt das genommene Medikament an
\item Der Benutzer drückt auf den Button mit der Aufschrift “Rücknahme”
\item Das System wechselt zur Darstellung eines einzelnen Medikaments, beschrieben im UseCase 1
}
\addfield{Ergebnis}{Die Einnahmebestätigung ist zurückgenommen. Die Erinnerung ist erneut zu bestätigen oder zu verschieben.}
\end{usecase}

\begin{usecase}
\addtitle{Usecase 4}{Mehrere Einnahmebestätigungen zurücknehmen}
\addfield{Primärer Akteur}{Patient}
\addfield{Vorbedingung}{Es wurde ein Medikationsplan aus der DB auf die Uhr geladen. Mehrere Medikationen, welche zur gleichen Zeit genommen wurden, wurde als genommen markiert}
\additemizedfield{Ablauf}{
\item Das System zeigt die genommenen Medikamente an
\item Der Benutzer drückt auf den Button mit der Aufschrift “Rücknahme”
\item Das System wechselt zur Darstellung mehrerer Medikamente, beschrieben im UseCase 2
}
\addfield{Ergebnis}{Die Einnahmebestätigung ist zurückgenommen. Die Erinnerung ist erneut zu bestätigen oder zu verschieben.}
\end{usecase}


Diese vier Usecases sind auch Grundlage für diese Arbeit. Jedoch werden nur die Usecases umgesetzt, welche auch in der Evaluation geprüft werden können.

\section{Apple`s Programmiersprache Swift}
Swift wurde im Juni 2014 von Apple vorgestellt und genießt seitdem steigendes Interesse. Im Juni 2015 wurde Version 2.0 veröffentlicht \cite{Apple:2014sp}. Mit Version 2.0 wurde ebenfalls der Plan vorgestellt, Swift Open Source zu machen und somit auch anderen Plattformen die Entwicklung mit Swift zu ermöglichen \cite{Apple:2014sp}. Diesen Plan setzte Apple Ende 2015 in die Tat um. Swift ist nun völlig Open Source. Es können Vorschläge für neue Sprachfunktionen gemacht werden. Auch Apple`s Entwicklungsteam diskutiert seine Pläne für die Sprache öffentlich \cite{Apple:2015swiftOpen}.

\subsection{Objective-C und Swift}
In den frühen 80er Jahren entwickelte Brad Cox die Sprache Objective-C\cite{Dalrymple:2009aa}. Weiter führt Dalrymple aus, dass die Sprache  die Vorteile einer schnellen C-Sprache mit den Vorteilen der objektorientierten Sprache SmallTalk verbinden sollte. Die Firma NextSTEP nutzte Objectiv-C und als NEXTStep von Apple aufgekauft wurde, integrierte Apple Objectiv-C und ermöglichte Mac-Entwicklern die Nutzung.

 Als Apple nun 2008 seine iOS Plattform öffnete und Entwickler eigene Anwendungen für das System schreiben konnten, bekam Objective-C neue Aufmerksamkeit. Viele Entwickler sahen Objective-C als ein Überbleibsel alter Zeiten und waren der Sprache gegenüber negativ eingestellt. Apple stand nun unter Zugzwang um seine Plattform, mit der große wirtschaftliche Interessen verbunden sind, für Entwickler attraktiv zu halten \cite{Wells:2015fu}. Da jedoch alle highlevel APIs in Objectiv-C vorhanden sind, ist es technisch nicht möglich, auf eine bekannte Sprache wie Java für iOS und Mac Entwicklung umzusteigen. Man entschied sich für eine Neuentwicklung mit Hinblick auf neue Programmierparadigmen und auf sehr gute Kompatibilität zu alten Objectiv-C APIs \cite{Wells:2015fu}.

\subsection{Überblick der Neuerung}
Swift bietet gegenüber Objectiv-C viele Neuerungen. Diese Neuerungen sind bekannte Konzepte, die schon in anderen Programmiersprachen Verwendung finden. Die meisten dieser Verbesserungen unterstützen den Entwickler früh, Fehler im Quellcode zu entdecken. Im folgenden sind die wichtigsten Neuerungen beschrieben.

\subsection*{Type Inference}
Bei Type Inference erkennt der Compiler, welcher Typ in eine Variable instanziiert wurde. Es ist nicht nötig, für die Variable eine Typendefinition anzugeben \cite{Apple:2014sp}. Hierdurch wird der Quellcode leichter lesbar. 
\lstinputlisting[caption=Beispiel zu Type Inference in Swift label=lst:typeInference]{02_stateArt/codeExamples/TypeInference.swift}

\subsection*{Closures - Functions as First Class Types}
\label{ch:closures}
Während bei strikt objektorientierten Sprachen nur Objekte und primitive Datentypen existieren, gibt es Sprachen, bei denen Funktionen als Typen existieren. Diese Funktionen können auch als Referenz in eine Variable gespeichert werden. Folgende Code Beispiele sollen dies veranschaulichen. Wir nutzen hierfür einen asynchronen Netzwerk-Request\cite{Apple:2014sp}.
\lstinputlisting[caption=Java Beispiel zu First Class Objects, label=lst:firstFuncJava]{02_stateArt/codeExamples/NetworkRequestWithoutFunctions.java}
 In Beispiel \ref{lst:firstFuncJava} wird Java als Repräsentant für eine strikt objektorientierte Sprache verwendet. Hier wird ein Interface definiert, welches der Nutzer des Netzwerk-Requests implementieren muss, um den Request zu empfangen. Beim Aufrufen des Requests muss nun der Aufrufer als Referenz übergeben werden, damit bei Abschließen des Requests der Aufrufer benachrichtigt werden kann (\lstinline{handleNetworkResponse}). 
\lstinputlisting[caption=Swift Beispiel zu First Class Objects, label=lst:firstFuncSwift]{02_stateArt/codeExamples/NetworkRequestWithFunctions.swift}
Im Codebeispiel \ref{lst:firstFuncSwift} benötigt es kein Interface für den Nutzer des Netzwerk-Requests. Es ist nun in Swift möglich, eine Funktion zu definieren und diese gleichzeitig in einer lokalen Variablen zu speichern. Nun kann diese Funktion als Referenz zum Netzwerk-Request übergeben werden. Die Funktion wird aufgerufen, wenn der Netzwerk-Request beendet ist. 

\subsection*{Generics}
Generics sind schon längere Zeit Teil moderner Programmiersprachen. Sie unterstützen den Entwickler, um Fehler zur Compilezeit zu entdecken. Ein sehr gutes Beispiel hierfür sind Collections (Arrays, Listen, Set, etc). Ein Collection Typ wie Array muss nicht für jeden Typ, den er hält, neu implementiert werden.
\lstinputlisting[caption=Swift Beispiel zu Generic Array Collection, label=lst:generics]{02_stateArt/codeExamples/Generics.swift}
Im Codebeispiel \ref{lst:generics} werden die Vorteile von Generics aufgezeigt. Der erste Versuch, einen Array zu implementieren (\lstinline{CustomMedicationArray}), zeigt, dass dieser Array eine sehr gute Typen Deklarierung enthält. Der Compiler kann den Entwickler also maximal unterstützen, jedoch ist diese Implementierung minimal wiederverwendbar. Der zweite Ansatz (\lstinline{CustomArray}) gibt als Typ-Einschränkung den globalen Supertyp an. Dies führt zu einer maximalen Wiederverwendbarkeit, da jede Klasse von diesem globalen Supertyp erbt (funktioniert nur theoretisch, da in Swift kein Zwang besteht, von einem globalen Supertyp zu erben).
Die letzte Implementierung (\lstinline{GenericArray}) nutzt nun Generics. So wird bei der Definition der Klasse ein generischer Typ \lstinline{ T}  eingeführt. Dieser Typ ist eine Art Platzhalter für einen konkreten Typ, der später vom Array gehalten wird. So muss bei der Initialisierung der generischen Klasse die Typ-Information für \lstinline{T} übergeben werden (siehe letze Zeile in \ref{lst:generics}).

Das oben beschriebene Beispiel dient nur zur Verdeutlichung des Konzeptes. Swift bietet eine Reihe an Collection Types, darunter auch eine generische Array Implementierung. 

\subsection*{Protocol Extension}
Protocol Extension helfen dem Entwickler Abstraktionen, die über ein Interface (in Swift Protocol) definiert werden, zu implementieren und dadurch Duplizierungen zu minimieren. So kann es sein, dass eine Methode eines Interfaces in jeder Klasse, die das Interface implementiert, gleich umgesetzt wird. Dies führt zu einer Code-Duplizierung. 
\lstinputlisting[caption=Swift Beispiel zu Generic Array Collection, label=lst:ProtocolExtensions]{02_stateArt/codeExamples/ProtocolExtensions.swift}
Im Codebeispiel \ref{lst:ProtocolExtensions} wird ein List Interface definiert. Die Extension zu diesem Interface ermöglicht die Implementierung von \lstinline{last}  und \lstinline{first}. Diese Implementierung teilen sich nun alle Klassen, die List implementieren.

\subsection*{Optional Types}
\label{ch:optionals}
Optional Types eröffnen neue Möglichkeiten beim Modellieren von Datenmodellen und Erstellen von APIs. Es ist so möglich, Argumente in einer Methode als \glqq 0ptional\grqq  zu kennzeichnen und somit dem Nutzer der Methode zu erlauben, eine null-Referenz zu übergeben. Ist kein Optional-Type gekennzeichnet, so verbietet der Compiler eine null-Referenz Übergabe \cite{Apple:2014sp}. Im folgenden Beispiel wird eine Medikation in Java ohne Optional Types und in Swift mit Optional Types modelliert.
\lstinputlisting[caption=Java Beispiel mit fehlenden Optional Types, label=lst:optinalsJava]{02_stateArt/codeExamples/OptionalType.java} 

Im Codebeispiel \ref{lst:optinalsJava} ist zu erkennen, dass zur Compilezeit keine Aussage über den Zustand der Instanz-Variablen getroffen werden kann. Der Entwickler muss also aus dem logischen Kontext erkennen, welche Variablen eine null-Referenz enthalten könnten. Dies kann zu Laufzeitfehlern führen.

\lstinputlisting[caption=Swift Beispiel mit Optional Types, label=lst:optinalsSwift]{02_stateArt/codeExamples/OptionalType.swift}
In \ref{lst:optinalsSwift} sind die Instanzvariablen nun mit Optional Types modelliert. Nun 
kann zur Compilezeit zugesichert werden, welche Variablen eine null-Referenz enthalten können und welche Variablen sicher mit einem Wert belegt sind. Dies führt zu weniger Fehlern während der Laufzeit.

Auch APIs können mit \glqq Optionals\grqq  modelliert werden. So darf ein Parameter nicht null sein, wenn er nicht als Optional definiert wurde. Dies macht eine API strikter und führt ebenfalls zu weniger Laufzeitfehlern, da Fehler schon zur Compilezeit erkannt werden.

\section{Wearables}
Übersetzt man Wearables ins Deutsche, bedeutet es \glqq Tragbares\grqq  oder \glqq Anziehbares\grqq  im Sinne von einem Kleidungsstück tragen. Wenn man nun den Begriff Wearables mit Kleidung assoziiert, liegt man nicht falsch. Anstatt Computer auf dem Schreibtisch oder in der Hosentasche zu haben, trägt man sie am Körper wie Kleidungs- oder Schmuckstücke \cite{Dvorak:2008aa}. Die Grenzen zwischen Kleidung und Computer verschmelzen und es ist manchmal nicht klar, was man schon als Wearable bezeichnen kann oder auch nicht.

Wearables sind meist mit Sensoren ausgestattet, die Daten aus der Körperregion sammeln, an der sie getragen werden \cite{4711366}. Diese Daten zeigen immer nur einen Teilausschnitt. Durch das Tragen von mehreren Geräten am Körper verteilt, können mehr Daten gesammelt werden. So ist es möglich, einen noch genaueren Überblick über den Zustand des Körpers zu erhalten\cite{4711366}.

Erst mit der Miniaturisierung der Computertechnik und Sensoren war es möglich, Computer mit integriertem Akku in Größe einer Streichholzschachtel herzustellen. Während Swatch 1995/1996 eine Uhr vorstellte, die als Skipass funktionierte, stellte Apple 18 Jahre später die Apple Watch vor, die über einen Mikroprozessor, WLAN und eine Vielzahl an Sensoren verfügt (mehr zur Apple Watch in \ref{ch:apple-watch})

Doch nicht nur Uhren zählen zu den Wearables. Google präsentierte mit der Google Glass eine Datenbrille, die über eine Kamera, ein Heads \glqq Prismatic head-mount\glqq  Display, Sprachsteuerung sowie Internetverbindung und weitere Sensoren verfügt \cite{Muensterer2014281}. Auch ein Ring, der mit Hilfe eines Sensors den Puls misst, existiert als Prototyp \cite{4711366}.

Während im Smartphone-Markt noch ein Focus auf Leistung und Funktion der Geräte gelegte wurde, darf man beim Wearables-Markt den Faktor der Ästhetik nicht vergessen. Hier wird ein Bereich betreten, der starke Einflüsse von Mode aufzeigt. Anwender eines Gerätes achten also nicht mehr nur auf die Funktion, sondern auch auf Form, Farbe und Lifestyle, den das Produkt verkörpert. Apple bietet unter dem Namen \glqq Watch Edition \grqq eine Apple Watch aus echtem Gold an, deren Preis über 10.000 Euro beträgt und sich an den Luxusmarkt richtet. Auch TAGHeuer, eine Firma, die sich auf luxuriöse und modische Uhren spezialisiert hat, betritt nun auch den Markt der Wearables \cite{TAGHeuer:20015aa}.

\section{Apple Watch}
\label{ch:apple-watch}
Eine Andorderung, welche sich aus dem Kontext dieser Arbeit entnehmen lässt, ist die Nutzung der Apple Watch als Zielplattform. Die Apple Watch wurde im September 2014 vorgestellt und startete im April 2015 mit dem Verkauf.
\subsection{Hardware}
Die Apple Watch existiert in zwei Versionen. Eine Uhr mit 38mm (272x340) und eine mit 42mm (312x390) großem Display. Die Uhr bietet einen 8GB großen internen Speicher. Mit einer Akkulaufzeit von 18h unter durchschnittlicher Nutzung hält die Uhr einen Tag durch \cite{Riches:2015aa}. 
\subsection{Abhängigkeit zum iPhone}
Die Apple Watch wurde als Erweiterung zum iPhone entwickelt. Und so sind auch viele integrale Funktionen nur vom iPhone aus steuerbar. Die Uhr kann ohne ein iPhone nicht durch den Setup-Prozess geleitet werden. Auch native Anwendungen (siehe \ref{ch:watch_software}) können nur über das iPhone installiert werden. Sind diese Schritte getan, also die Uhr betriebsbereit und Anwendungen installiert, kann die Uhr teilweise auch autonom agieren. So kann sie auch ohne iPhone, über WLAN, mit dem Internet kommunizieren.

\subsection{Software}
\label{ch:watch_software}
Mit Erscheinen der Uhr wurde auch das Betriebssystem in Version 1 ausgeliefert und dazu das \gls{sdk} namens WatchKit. Dies erlaubte es Anwendungen zu entwickeln, welche auf dem verbundenen iPhone ausgeführt wurden. Die Folge war schlechter Performance der Anwendungen und zu völliger Abhängigkeit zum iPhone.

Im Juni 2015 veröffentlichte Apple die erste Vorabversion von watchOS 2.0, welches später im September 2015 für Endnutzer bereit gestellt wurde. WatchOS bietet mehr Unabhängigkeit für Anwendungen, da diese direkt auf der Uhr ausgeführt werden. Für Anwendungsentwickler gibt es vier Arten, Informationen auf der Uhr darzustellen. Es handelt sich um native Anwendungen (Apps), Glances, Complications und Actionable Notifications \cite{Apple:2015devAw}.

Native Anwendungen sind fest installiert auf der Uhr. Sie können unabhängig auf der Uhr gestartet werden. In einer nativen Anwendung lassen sich komplexere Programme realisieren, da der Nutzer durch viele Möglichkeiten zur Eingabe und Interaktion (\ref{ch:eingabe_interface}) hat. Installiert werden die Apps vom iPhone aus. Eine Watch-App benötigt immer eine iPhone App, die jeweils auf dem iPhone installiert ist.

Complications sind kleine Interface Elemente, die sich auf dem Zifferblatt der Uhr platzieren lassen. So können mit einem Blick auf die Uhrzeit auch Informationen aus der App abgelesen werden.

Ein Glance ist ein Interface, auf dem die wichtigsten Informationen einer App übersichtlich dargestellt werden. Der Nutzer soll mit einem Blick die Informationen erkennen. Es ist keine Interaktion mit einem Glance möglich. Tippt der Nutzer auf einen Glance, öffnet sich die zugehörige App. Glances sind optional zu einer App zu entwickeln.

Notifications, oder auch auf deutsch Benachrichtigungen, informieren den Benutzer über Ereignisse. Diese Ereignisse können entweder zeitlich geplant sein, vom Betreten einer Ortskoordinate ausgelöst werden, oder von einem Server auf das Gerät gepusht (Server sendet ein Ereignis, wie z.B. eine Nachricht) werden. Notifications können Aktionen beinhalten. So kann eine Bestätigung einer Notification direkt geschehen, ohne dafür die dazugehörige Anwendung zu starten. Dies nennt man eine Actionable Notification \cite{Apple:2015devAw}.

Es gibt zwei verschiedene Arten von Notifications, die beide die gleiche Funktion haben, jedoch mit unterschiedlichem Informationsgehalt angereichert werden. Zum Einen die Standard Notification. Diese Notification wird vom iPhone gespiegelt. Sie bietet nicht mehr Informationen gegenüber der iPhone Notification. Es ist jedoch möglich, mit einer nativen Anwendung auch eine optimale Darstellung der Notification zu entwickeln. Diese Darstellung kann detaillierte Informationen beinhalten wie Bilder, Karten oder eine genauere Beschreibung der Information\cite{Apple:2015notif}.

\subsection{Schnittstellen}
Bluetooth 4.0 und Wi-Fi 802.11b/g sind die Netzwerkschnittstellen der Apple Watch. Dazu kommt noch ein NFC Chip, der jedoch nicht über eine API nutzbar ist und vorerst nur für Apple Pay, dem Apple eigenen Bezahldienst, vorgesehen ist\cite{RITCHIE:2015aa}. 
\subsection{Sensoren}
Die Apple Watch besitzt einen Beschleunigungssensor und Gyroskop, welche genaue Bewegungsdaten liefern. Ebenso ist ein optischer Herzschlagsensor verbaut, der an der Unterseite der Uhr auf der Haut anliegt. Ein Mikrofon, welches für Spracheingabe genutzt werden kann, ist auch vorhanden.

\subsection{Eingabe Interfaces}
\label{ch:eingabe_interface}
Neben eines normalen Touchscreens führte Apple in der Uhr auch eine Eingabeart namens Force Touch ein. Diese Technologie erlaubt der Uhr zu erkennen, wie fest der Nutzer auf das Display drückt. Dies ermöglicht eine neue Art der Eingabe. Besonders bei einer kleinen Eingabefläche, wie die der Apple Watch, ist eine neue Interaktionsdimension interessant, da es eine differenziertere Interaktion erlaubt. Leider ist eine mögliche Interaktion mit Force Touch optisch nicht zu erkennen, was eine klare Nutzerführung schwer macht.

 Von analogen Uhren ist die Krone, das Rad an der Seite einer Uhr, bekannt. An ihr kann die Uhr eingestellt oder aufgezogen werden. An der Apple Watch ist die Krone auch zu finden. Apple nennt sie \glqq Digital Crown\grqq, also digitale Krone. Sie befindet sich ebenfalls an der Seite der Uhr. Die Krone ist drehbar und lässt sich als Druckknopf nutzen. Sie dient zum Scrollen von Inhalten sowie zum präzisem Auswählen von Elementen aus einer Auswahlliste. Durch das Nutzen der Krone wird der Bildschirm nicht durch einen Finger verdeckt, was bei einem kleinen Display von großem Vorteil ist.
\subsection{Armband}
Apple bietet sechs verschiedene Armbänder für die Apple Watch an (Stand Nov. 2015). Zusätzlich ist es möglich, Armbänder von Drittherstellern zu erwerben. 

Das in der günstigsten Version mitgelieferte Armband (Sportarmband) verfügt über einen sehr komplizierten Verschlussmechanismus und ist deswegen weniger für Menschen geeignet, die über schwache sensomotorische Fähigkeiten verfügen. Es gibt auch Armbänder, die einen magnetischen Verschluss bieten und deswegen leichter zu handhaben sind. Diese Armbänder sind jedoch drei mal so teuer wie das Sportarmband.

An der Verbindung zwischen Armband und Uhr ist ein nicht weiter spezifizierter Wartungsport verbaut. Dieser Anschluss könnte in Zukunft Armbänder ermöglichen, die Informationen aus dem Armband an die Uhr zu übermitteln.
\subsection{Prototyping}
Zum Erstellen des frühen Prototyps wurden Papier-Prototypen erstellt. Durch die Nutzung dieses Werkzeuges konnte schnell und iterativ gearbeitet werden. Da an dieser Arbeit nur eine Person gearbeitet hat, bietet sich diese Methode an. Sie ist schnell zu erlernen, setzt keine Vorkenntnisse voraus und erzielt schnell Ergebnisse.


