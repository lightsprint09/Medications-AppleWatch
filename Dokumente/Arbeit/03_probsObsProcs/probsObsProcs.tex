%%
% ----------------------------------------------------------------------------
% "THE BEER-WARE LICENSE" (Revision 42):
% <sebastian.rauh@hs-heilbronn.de/michael.bauer@hs-heilbronn.de> wrote this 
% file. As long as you retain this notice you can do whatever you want with 
% this stuff. If we meet some day, and you think this stuff is worth it, you 
% can buy us a beer in return. 
% Michael Lukas Bauer, Sebastian Felix Rauh
% ----------------------------------------------------------------------------
%%
In diesem Kapitel werden Ergebnisse der Analyse und des Entwurfes erläutert. Große Teile der Anforderungsanalyse stammen aus dem \gls{pita}.
\section{Entwicklungsmethodik}
Da das Projekt nur durch eine Person durchgeführt wurde, kann man nicht von einer definierbaren Methodik sprechen. Es handelte sich um ein iteratives Vorgehen. Funktionen, die bei der Evaluation erarbeitet wurden, sind direkt in eine neue Version der Anforderungen integriert worden. Der Quellcode wurde mit \gls{git} versioniert verwaltet.
\section{Anforderungsanalyse}
Die Anforderungen wurden von einem Gespräch mit Monika Pobiruchin am Anfang vom \gls{pita} erhoben. Kernaussage dieses Gesprächs war, dass das System von alten Menschen genutzt werden soll, um Medikamente regelmäßiger einzunehmen. Dies beinhaltet eine geringe Auseinandersetzung mit der Technik des Systems. Die Uhr soll möglichst autonom sein und nicht zwingend an ein Smartphone gekoppelt sein.

\section{Usecases} 
Von dem Gespräch mit Monika Pobiruchin wurde eine Persona für die Zielgruppe erstellt.
Mit Hilfe der Persona wurden die Kern-Usecases abgeleitet. Die folgenden Usecases 1 bis 4 wurden im \gls{pita} erarbeitet und von dort übernommen. 
\begin{usecase}
\addtitle{Usecase 1}{Der Patient wird an ein Medikament erinnert}
\addfield{Primärer Akteur}{Patient}
\addfield{Beschreibung}{Ein Erinnerungs Pop-Up erscheint auf dem Display und es wird ein Signal/eine Vibration ausgelöst. Das Pop-Up zeigt ein Abbild des Medikaments, dessen Namen und die Uhrzeit, zu der es eingenommen werden soll.}
\addfield{Vorbedingung}{Es wurde ein Medikationsplan aus der DB auf die Uhr geladen.}
\additemizedfield{Ablauf}{
\item Das Pop-Up mit der Erinnerung erscheint auf dem Smartwatch-Display. Die Erinnerung beinhaltet Informationen zur Uhrzeit, Menge und Art der Medikation
\item Der Patient bestätigt, dass er das Medikament genommen hat
\item Gerät bestätigt visuell, dass der Patient das Medikament als genommen markiert hat
}
\additemizedfield{Alternativablauf}{
\item Das Pop-Up mit der Erinnerung erscheint auf dem Smartwatch-Display
\item Der Patient wählt Option zum Verschieben der Medikation aus
\item Auf einem zusätzlichen Dialog kann er aus einer Auswahl eine Zeitdauer wählen, um die die Medikation verschoben wird}
\additemizedfield{Alternativablauf 2}{
\item Das Pop-Up mit der Erinnerung erscheint auf dem Smartwatch-Display
\item Der Patient reagiert nicht auf die Erinnerung
\item Die Erinnerung wird alle x Minuten wiederholt, solange der Patient nicht reagiert.
\item Das Fehlen einer Reaktion des Patientens innerhalb einer Zeit von x Minuten wird vermerkt}
\addfield{Ergebnis}{Der Patient hat die Einnahme des Medikaments bestätigt und dieses auch eingenommen}
\addfield{Alternativergebnis 1}{Der Patient hat die Erinnerung an die Medikamenteneinnahme verschoben}
\addfield{Alternativergebnis 2}{Der Patient hat die Erinnerung an das Medikament ausgeschaltet}
\end{usecase}

\begin{usecase}
\addtitle{Usecase 2}{Der Patient wird an mehrere Medikamente erinnert}
\addfield{Primärer Akteur}{Patient}
\addfield{Beschreibung}{Ein Erinnerungs Pop-Up erscheint auf dem Display und es wird ein Signal/eine Vibration ausgelöst. Das Pop-Up zeigt eine Liste der Medikamente, die eingenommen werden müssen.}
\addfield{Vorbedingung}{Es wurde ein Medikationsplan aus der DB auf die Uhr geladen.}
\additemizedfield{Ablauf}{
\item Das Pop-Up mit der Erinnerung erscheint auf dem Smartwatch-Display. Die Erinnerung beinhaltet Informationen zur Uhrzeit, Menge und Art der Medikationen
\item Der Patient bestätigt, dass er die Medikamente alle genommen hat
\item Gerät bestätigt visuell, dass der Patient die Medikamente als genommen markiert hat
}
\additemizedfield{Alternativablauf}{
\item Das Pop-Up mit der Erinnerung erscheint auf dem Smartwatch-Display
\item Der Patient wählt Option zum Verschieben der Medikationen aus
\item Auf einem zusätzlichem Dialog kann er aus einer Auswahl eine Zeitdauer wählen, um das die Medikationen verschoben werden}
\additemizedfield{Alternativablauf 2}{
\item Das Pop-Up mit der Erinnerung erscheint auf dem Smartwatch-Display
\item Der Patient wählt ein Medikament von der Liste aus
\item Patient befindet sich nun im Usecase 1}
\additemizedfield{Alternativablauf 3}{
\item Das Pop-Up mit der Erinnerung erscheint auf dem Smartwatch-Display
\item Der Patient reagiert nicht auf die Erinnerung
\item Die Erinnerung wird alle x Minuten wiederholt, solange der Patient nicht reagiert.
\item Das Fehlen einer Reaktion des Patientens innerhalb einer Zeit von x Minuten wird vermerkt}
\addfield{Ergebnis}{Der Patient hat die Einnahme der Medikamente bestätigt und diese auch eingenommen}
\addfield{Alternativergebnis 1}{Der Patient hat die Erinnerung an die Medikamenteneinnahme verschoben}
\addfield{Alternativergebnis 2}{Der Patient hat die Erinnerung an das Medikament ausgeschaltet}
\addfield{Alternativergebnis 3}{Der Patient hat bestimmte Medikamente ausgewählt und mit dem weiterführenden Usecase 1 bearbeitet}
\end{usecase}

\begin{usecase}
\addtitle{Usecase 3}{Einzelne Einnahmebestätigung zurücknehmen}
\addfield{Primärer Akteur}{Patient}
\addfield{Vorbedingung}{Es wurde ein Medikationsplan aus der DB auf die Uhr geladen. Eine Medikation wurde als genommen markiert}
\additemizedfield{Ablauf}{
\item Das System zeigt das genommene Medikament an
\item Der Benutzer drückt auf den Button mit der Aufschrift “Rücknahme”
\item Das System wechselt zur Darstellung eines einzelnen Medikaments, beschrieben im UseCase 1
}
\addfield{Ergebnis}{Die Einnahmebestätigung ist zurückgenommen. Die Erinnerung ist erneut zu bestätigen oder zu verschieben.}
\end{usecase}

\begin{usecase}
\addtitle{Usecase 4}{Mehrere Einnahmebestätigungen zurücknehmen}
\addfield{Primärer Akteur}{Patient}
\addfield{Vorbedingung}{Es wurde ein Medikationsplan aus der DB auf die Uhr geladen. Mehrere Medikationen, welche zur gleichen Zeit genommen wurden, wurde als genommen markiert}
\additemizedfield{Ablauf}{
\item Das System zeigt die genommenen Medikamente an
\item Der Benutzer drückt auf den Button mit der Aufschrift “Rücknahme”
\item Das System wechselt zur Darstellung mehrerer Medikamente, beschrieben im UseCase 2
}
\addfield{Ergebnis}{Die Einnahmebestätigung ist zurückgenommen. Die Erinnerung ist erneut zu bestätigen oder zu verschieben.}
\end{usecase}


Diese vier Usecases sind auch Grundlage für diese Arbeit. Wie die Usecases umgesetzt sind, wird in Kapitel \ref{ch:realisation} beschrieben.

\section{Apple Watch}
\label{ch:apple-watch}
Eine Andorderung, welche sich aus dem Kontext dieser Arbeit entnehmen lässt, ist die Nutzung der Apple Watch als Zielplattform. Die Apple Watch wurde im September 2014 vorgestellt und startete im April 2015 mit dem Verkauf.
\subsection{Hardware}
Die Apple Watch existiert in zwei Versionen. Eine Uhr mit 38mm (272x340) und eine mit 42mm (312x390) großem Display. Die Uhr bietet einen 8GB großen internen Speicher. Mit einer Akkulaufzeit von 18h unter durchschnittlicher Nutzung hält die Uhr einen Tag durch \cite{Riches:2015aa}. 
\subsection{Abhängigkeit zum iPhone}
Die Apple Watch wurde als Erweiterung zum iPhone entwickelt. Und so sind auch viele integrale Funktionen nur vom iPhone aus steuerbar. Die Uhr kann ohne ein iPhone nicht durch den Setup-Prozess geleitet werden. Auch native Anwendungen (siehe \ref{ch:watch_software}) können nur über das iPhone installiert werden. Sind diese Schritte getan, also die Uhr betriebsbereit und Anwendungen installiert, kann die Uhr teilweise auch autonom agieren. So kann sie auch ohne iPhone, über WLAN, mit dem Internet kommunizieren.

\subsection{Software}
\label{ch:watch_software}
Mit Erscheinen der Uhr wurde auch das Betriebssystem in Version 1 ausgeliefert und dazu das \gls{sdk} namens WatchKit. Dies erlaubte es Anwendungen zu entwickeln, welche auf dem verbundenen iPhone ausgeführt wurden. Dies führte zu schlechter Performance der Anwendungen und zu völliger Abhängigkeit zum iPhone.

Im Juni 2015 veröffentlichte Apple die erste Vorabversion von watchOS 2.0, welches später im September 2015 für Endnutzer bereit gestellt wurde. WatchOS bietet mehr Unabhängigkeit für Anwendungen, da diese direkt auf der Uhr ausgeführt werden. Für Anwendungsentwickler gibt es vier Arten Informationen auf der Uhr darzustellen. Es handelt sich um native Anwendungen (Apps), Glances, Complications und Actionable Notifications \cite{Apple:2015devAw}.

Native Anwendungen sind fest installiert auf der Uhr. Sie können unabhängig auf der Uhr gestartet werden. In einer nativen Anwendung lassen sich komplexere Programme realisieren, da der Nutzer durch viele Möglichkeiten zur Eingabe und Interaktion (\ref{ch:eingabe_interface}) hat. Installiert werden die Apps vom iPhone aus. Eine Watch-App benötigt immer eine iPhone App, die jeweils auf dem iPhone installiert ist.

Complications sind kleine Interface Elemente, die sich auf dem Zifferblatt der Uhr platzieren lassen. So können mit einem Blick auf die Uhrzeit auch Informationen aus der App abgelesen werden.

Ein Glance ist ein Interface, auf dem die wichtigsten Informationen einer App übersichtlich dargestellt werden. Der Nutzer soll mit einem Blick die Informationen erkennen. Es ist keine Interaktion mit einem Glance möglich. Tippt der Nutzer auf einen Glance, öffnet sich die zugehörige App. Glances sind optional zu einer App zu entwickeln.

Notifications, oder auch auf deutsch Benachrichtigungen, informieren den Benutzer über Ereignisse. Diese Ereignisse können entweder zeitlich geplant sein, vom Betreten einer Ortskoordinate ausgelöst werden, oder von einem Server auf das Gerät gepusht (Server sendet ein Ereignis, wie z.B. eine Nachricht) werden. Notifications können Aktionen beinhalten. So kann eine Bestätigung einer Notification direkt geschehen, ohne dafür die dazugehörige Anwendung zu starten. Dies nennt man eine Actionable Notification \cite{Apple:2015devAw}.

Es gibt zwei verschiedene Arten von Notifications, die beide die gleiche Funktion haben, jedoch mit unterschiedlichem Informationsgehalt angereichert werden. Zum Einen die Standard Notification. Diese Notification wird vom iPhone gespiegelt. Sie bietet nicht mehr Informationen gegenüber der iPhone Notification. Es ist jedoch möglich, mit einer nativen Anwendung auch eine optimale Darstellung der Notification zu entwickeln. Diese Darstellung kann detaillierte Informationen beinhalten wie Bilder, Karten oder eine genauere Beschreibung der Information\cite{Apple:2015notif}.

\subsection{Schnittstellen}
Bluetooth 4.0 und Wi-Fi 802.11b/g sind die Netzwerkschnittstellen der Apple Watch. Dazu kommt noch ein NFC Chip, der jedoch nicht über eine API nutzbar ist und vorerst nur für Apple Pay, dem Apple eigenen Bezahldienst, vorgesehen ist\cite{RITCHIE:2015aa}. 
\subsection{Sensoren}
Die Apple Watch besitzt einen Beschleunigungssensor und Gyroskop, welche genaue Bewegungsdaten liefern. Ebenso ist ein optischer Herzschlagsensor verbaut, der an der Unterseite der Uhr auf der Haut anliegt. Ein Mikrofon, welches für Spracheingabe genutzt werden kann, ist auch vorhanden.

\subsection{Eingabe Interfaces}
\label{ch:eingabe_interface}
Neben eines normalen Touchscreens führte Apple in der Uhr auch eine Eingabeart namens Force Touch ein. Diese Technologie erlaubt der Uhr zu erkennen, wie fest der Nutzer auf das Display drückt. Dies ermöglicht eine neue Art der Eingabe. Besonders bei einer kleinen Eingabefläche, wie die der Apple Watch, ist eine neue Interaktionsdimension interessant, da es eine differenziertere Interaktion erlaubt. Leider ist eine mögliche Interaktion mit Force Touch optisch nicht zu erkennen, was eine klare Nutzerführung schwer macht.

 Von analogen Uhren ist die Krone, das Rad an der Seite einer Uhr, bekannt. An ihr kann die Uhr eingestellt oder aufgezogen werden. An der Apple Watch ist die Krone auch zu finden. Apple nennt sie \glqq Digital Crown\grqq, also digitale Krone. Sie befindet sich ebenfalls an der Seite der Uhr. Die Krone ist drehbar und lässt sich ebenfalls als Druckknopf nutzen. Sie dient zum Scrollen von Inhalten, sowie zum präzisem Auswählen von Elementen aus einer Auswahlliste. Durch das Nutzen der Krone wird der Bildschirm nicht durch einen Finger verdeckt, was bei einem kleinen Display von großem Vorteil ist.
\subsection{Armband}
Apple bietet sechs verschiedene Armbänder für die Apple Watch an (Stand Nov. 2015). Zusätzlich ist es möglich, Armbänder von Drittherstellern zu erwerben. 

Das in der günstigsten Version mitgelieferte Armband (Sportarmband) verfügt über einen sehr komplizierten Verschlussmechanismus und ist deswegen weniger für Menschen geeignet, die über schwache sensomotorische Fähigkeiten verfügen. Es gibt auch Armbänder, die einen magnetischen Verschluss bieten und deswegen leichter zu handhaben sind. Diese Armbänder sind jedoch drei mal so teuer wie das Sportarmband.

An der Verbindung zwischen Armband und Uhr ist ein nicht weiter spezifizierter Wartungsport verbaut. Dieser Anschluss könnte in Zukunft Armbänder ermöglichen, die Informationen aus dem Armband an die Uhr übermitteln.
\subsection{Prototyping}
Zum Erstellen des frühen Prototyps wurden Papier-Prototypen erstellt. Durch die Nutzung dieses Werkzeuges konnte schnell und iterativ gearbeitet werden. Da an dieser Arbeit nur eine Person gearbeitet hat, bietet sich diese Methode an. Sie ist schnell zu erlernen, setzt keine Vorkenntnisse voraus und erzielt schnell Ergebnisse.


