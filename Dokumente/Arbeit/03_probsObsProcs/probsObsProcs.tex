%%
% ----------------------------------------------------------------------------
% "THE BEER-WARE LICENSE" (Revision 42):
% <sebastian.rauh@hs-heilbronn.de/michael.bauer@hs-heilbronn.de> wrote this 
% file. As long as you retain this notice you can do whatever you want with 
% this stuff. If we meet some day, and you think this stuff is worth it, you 
% can buy us a beer in return. 
% Michael Lukas Bauer, Sebastian Felix Rauh
% ----------------------------------------------------------------------------
%%
In diesem Kapitel Ergebnisse der Analyse und des Entwurfes erläuteret. große Teile der Anforderungsanalyse stammen aus dem PITA Praktikum.
\section{Entwicklungsmethodik}
Da das Projekt nur durch eine Person durchgeführt wurde, kann man nicht von einem definierbaren Methodik sprechen. Es handelte sich um ein iteratives Vorgehen. Funktionen, die bei der Evaluation erarbeitet wurden, sind direkt in eine neue Version der Anforderungen integriert worden. Der Quellcode wurde mit \gls{git} versioniert verwaltet
\section{Anforderungsanalyse}
  