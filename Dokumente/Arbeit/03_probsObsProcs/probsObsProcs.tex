%%
% ----------------------------------------------------------------------------
% "THE BEER-WARE LICENSE" (Revision 42):
% <sebastian.rauh@hs-heilbronn.de/michael.bauer@hs-heilbronn.de> wrote this 
% file. As long as you retain this notice you can do whatever you want with 
% this stuff. If we meet some day, and you think this stuff is worth it, you 
% can buy us a beer in return. 
% Michael Lukas Bauer, Sebastian Felix Rauh
% ----------------------------------------------------------------------------
%%
In diesem Kapitel Ergebnisse der Analyse und des Entwurfes erläuteret. große Teile der Anforderungsanalyse stammen aus dem PITA Praktikum.
\section{Entwicklungsmethodik}
Da das Projekt nur durch eine Person durchgeführt wurde, kann man nicht von einem definierbaren Methodik sprechen. Es handelte sich um ein iteratives Vorgehen. Funktionen, die bei der Evaluation erarbeitet wurden, sind direkt in eine neue Version der Anforderungen integriert worden. Der Quellcode wurde mit \gls{git} versioniert verwaltet
\section{Anforderungsanalyse}
Die Anforderungen wurden von eine Gerspräch mit Monika Pobiruchin am Anfang des Pita-Praktikum erhoben. Kernaussage dieser Anforderungen, dass das System von alten Menschen genutzt werden kann, um Medikamente regelmäßiger einzunehmen. Dies beinhaltet eine geringe Auseinandersetzung mit der Technik des Systems. Die Uhr soll möglichst autonom sein und nicht zwingen an ein Smartphone gekoppelt sein.

\section{Usecases} 
Von dem Gespräch mit Monika Pobiruchin wurde eine Persona für die Zielgruppe abgeleitet. Diese Persona ist im Anhang zu finden.
Mit Hilfe der Persona, wurden die Kern-Usacases abgeleitet. Die folgenden Usecase 1 bis 4 wurden im Praktikum erarbeitet und wurden von dort übernommen. 
\begin{usecase}
\addtitle{Usecase 1}{Der Patient wird an ein Medikament erinnert}
\addfield{Primärer Akteur}{Patient}
\addfield{Beschreibung}{Ein Erinnerungs Pop-Up erscheint auf dem Display und es wird ein Signal/eine Vibration ausgelöst. Das Pop-Up zeigt ein Abbild des Medikaments, dessen Namen und die Uhrzeit, zu der es eingenommen werden soll.}
\addfield{Vorbedingung}{Es wurde ein Medikationsplan aus der DB auf die Uhr geladen.}
\additemizedfield{Ablauf}{
\item Das Pop-Up mit der Erinnerung erscheint auf dem Smartwatch-Display. Die Erinnerung beinhaltet Informationen zur Uhrzeit, Menge und Art der Medikation
\item Der Patient bestätigt, dass er das Medikament genommen hat
\item Gerät bestätigt visuell, dass der Patient das Medikament als genommen markiert hat
}
\additemizedfield{Alternativablauf}{
\item Das Pop-Up mit der Erinnerung erscheint auf dem Smartwatch-Display
\item Der Patient wählt Option zum Verschieben der Medikation aus
\item Auf einem zusätzlichen Dialog kann er aus einer Auswahl eine Zeitdauer wählen, um die die Medikation verschoben wird}
\additemizedfield{Alternativablauf 2}{
\item Das Pop-Up mit der Erinnerung erscheint auf dem Smartwatch-Display
\item Der Patient reagiert nicht auf die Erinnerung
\item Die Erinnerung wird alle x Minuten wiederholt, solange der Patient nicht reagiert.
\item Das Fehlen einer Reaktion des Patientens innerhalb einer Zeit von x Minuten wird vermerkt}
\addfield{Ergebnis}{Der Patient hat die Einnahme des Medikaments bestätigt und dieses auch eingenommen}
\addfield{Alternativergebnis 1}{Der Patient hat die Erinnerung an die Medikamenteneinnahme verschoben}
\addfield{Alternativergebnis 2}{Der Patient hat die Erinnerung an das Medikament ausgeschaltet}
\end{usecase}

\begin{usecase}
\addtitle{Usecase 2}{Der Patient wird an mehrere Medikamente erinnert}
\addfield{Primärer Akteur}{Patient}
\addfield{Beschreibung}{Ein Erinnerungs Pop-Up erscheint auf dem Display und es wird ein Signal/eine Vibration ausgelöst. Das Pop-Up zeigt eine Liste der Medikamente, die eingenommen werden müssen.}
\addfield{Vorbedingung}{Es wurde ein Medikationsplan aus der DB auf die Uhr geladen.}
\additemizedfield{Ablauf}{
\item Das Pop-Up mit der Erinnerung erscheint auf dem Smartwatch-Display. Die Erinnerung beinhaltet Informationen zur Uhrzeit, Menge und Art der Medikationen
\item Der Patient bestätigt, dass er die Medikamente alle genommen hat
\item Gerät bestätigt visuell, dass der Patient die Medikamente als genommen markiert hat
}
\additemizedfield{Alternativablauf}{
\item Das Pop-Up mit der Erinnerung erscheint auf dem Smartwatch-Display
\item Der Patient wählt Option zum Verschieben der Medikationen aus
\item Auf einem zusätzlichem Dialog kann er aus einer Auswahl eine Zeitdauer wählen, um das die Medikationen verschoben werden}
\additemizedfield{Alternativablauf 2}{
\item Das Pop-Up mit der Erinnerung erscheint auf dem Smartwatch-Display
\item Der Patient wählt ein Medikament von der Liste aus
\item Patient befindet sich nun im Usecase 1}
\additemizedfield{Alternativablauf 3}{
\item Das Pop-Up mit der Erinnerung erscheint auf dem Smartwatch-Display
\item Der Patient reagiert nicht auf die Erinnerung
\item Die Erinnerung wird alle x Minuten wiederholt, solange der Patient nicht reagiert.
\item Das Fehlen einer Reaktion des Patientens innerhalb einer Zeit von x Minuten wird vermerkt}
\addfield{Ergebnis}{Der Patient hat die Einnahme der Medikamente bestätigt und diese auch eingenommen}
\addfield{Alternativergebnis 1}{Der Patient hat die Erinnerung an die Medikamenteneinnahme verschoben}
\addfield{Alternativergebnis 2}{Der Patient hat die Erinnerung an das Medikament ausgeschaltet}
\addfield{Alternativergebnis 3}{Der Patient hat bestimmte Medikamente ausgewählt und mit dem weiterführenden Usecase 1 bearbeitet}
\end{usecase}

\begin{usecase}
\addtitle{Usecase 3}{Einzelne Einnahmebestätigung zurücknehmen}
\addfield{Primärer Akteur}{Patient}
\addfield{Vorbedingung}{Es wurde ein Medikationsplan aus der DB auf die Uhr geladen. Eine Medikation wurde als genommen markiert}
\additemizedfield{Ablauf}{
\item Das System zeigt das genommene Medikament an
\item Der Benutzer drückt auf den Button mit der Aufschrift “Rücknahme”
\item Das System wechselt zur Darstellung eines einzelnen Medikaments, beschrieben im UseCase 1
}
\addfield{Ergebnis}{Die Einnahmebestätigung ist zurückgenommen. Die Erinnerung ist erneut zu bestätigen oder zu verschieben.}
\end{usecase}

\begin{usecase}
\addtitle{Usecase 4}{Mehrere Einnahmebestätigungen zurücknehmen}
\addfield{Primärer Akteur}{Patient}
\addfield{Vorbedingung}{Es wurde ein Medikationsplan aus der DB auf die Uhr geladen. Mehrere Medikationen, welche zur gleichen Zeit genommen wurden, wurde als genommen markiert}
\additemizedfield{Ablauf}{
\item Das System zeigt die genommenen Medikamente an
\item Der Benutzer drückt auf den Button mit der Aufschrift “Rücknahme”
\item Das System wechselt zur Darstellung mehrerer Medikamente, beschrieben im UseCase 2
}
\addfield{Ergebnis}{Die Einnahmebestätigung ist zurückgenommen. Die Erinnerung ist erneut zu bestätigen oder zu verschieben.}
\end{usecase}


Diese 4 Usecases sind auch Grundlage für dies Arbeit. Wie die Usecases umgesetzt sind wird in Kapitel \ref{ch:realisation} beschrieben. In Kapitel \ref{ch:summ-eva-outl} finden sich überarbeitet Usecases, die Verbesserungen enthalten, welche aus der Evaluation mit der Zielgruppe hervorgehen.

\section{Apple Watch}
Eine Andorderung, welche sich aus dem Kontext dieser Arbeit entnehmen lässt, ist die Nutzung der Apple Watch als Zielplatform. Die Apple Watch wurde im September 2014 vorgestellt und staret im April 2015 mit dem Verkauf.
\subsection{Hardware}
Die Apple Watch existiert in zwei Versionen. Einen Uhr mit 38mm (272x340) und eine mit 42mm (312x390) großem Display. Die bietet einen 8GB großen internen Speicher. Mit einer Akkulaufzeit von 18h unter durchschnittlicher Nutzung, hält die Uhr einen Tag durch \cite{Riches:2015aa}. 
\subsection{Software}
Mit erscheinen der Uhr wurde auch das Betriebsystem in Version 1 ausgeliefert und dazu das \gls{sdk} Names WatchKit. Dies erlaubte es Entwicklern Anwendungen zu Entwickeln, welche auf dem verbunden iPhone ausgeführt wurden. Diese führte zu schlechte Performance der Anwendungen und zu vollen Abhängigkeit zum iPhone.

Im Juni veröffentlichte Apple die erste Vorabversion von watchOS 2.0, welches später im September 2015 für Endnutzer bereit gestellt wurde. watchOS biete mehr Unabhängigkeit für Anwendungen. Die Andwendugen laufen direkt auf der Uhr.
\subsection{Schniststellen}
Bluetooth 4.0 und Wi-Fi 802.11b/g sind die Netzwerkschnistellen der Apple Watch. Dazu kommt noch ein NFC Chip, der jedoch nicht über eine API nutzbar ist und vorerst nur für Apple Pay, dem Apple eigenen Bezahldienst, vorgesehen ist\cite{RITCHIE:2015aa}. 
\subsection{Sensoren}
Die Apple Watch besitzt einen Beschleunigungssensor und Gyroskop welche genaue Bewegungsdaten liefern. Ebenso ist ein optischer Herzschagsensor verbaut, der an der Unterseite der Uhr auf der Haut anliegt. Ein Mikrofon, welches für Spracheingabe genutzt werden kann ist auch vorhanden.
\subsection{Eingabe Interfaces}
Neben eines normalen Touchescreen führte Apple in der Uhr auch eine Eingabeart Names Force Touch ein. Diese Technologie erlaubt es der Uhr zu erkennen, wie fest der Nutzer auf das Display drückt. Dies ermöglicht eine Neue Art der Eingabe. Besonderns bei einer kleinen Eingabefläche, wie die der Apple Watch, ist eine neue Interaktionsdimesion interessant, da es eine differenziertere Interaktion erlaubt. Leider ist eine mögliche Interaktion mit Force Touch optisch nicht zu erkennen, was eine klare Nutzerführung schwer macht.

 Von Analogen Uhren ist die Krone, also das Rad an der Seite einer Uhr, an dem die Uhr eingestellt oder aufgezogen werden kann bekannt. An der Apple Watch ist die Krone auch zu finden. Apple nennt sie "Digital Crown", also digitale Krone. Sie befindet sich ebenfalls an der Seite der Uhr. Sie dient zum scrollen von Inhalten, sowie präzisen Auswählen von Elementen aus einer Auswahlliste. Durch nutzen der Krone wird der Bildschirm nicht durch einen Finger verdeckt, was bei einem kleinen Display von Vorteil ist.

\subsection{Armband}
Apple biete 6 verschiede Armbänder für die Apple Watch an (Stand Nov. 2015). Zusätzlich ist es möglich, Armbänder von Drittherstellern zu erwerben. 

Das in der günstigsten Version mitgelieferter Version (Sportarmband) verfügt über einen sehr komplizierten Verschlussmechanismus und ist deswegen weniger für Menschen geeignet, die über schwache sensomotorische Fähigkeiten verfügen. Es gibt auch Armbänder, die einen magnetischen Verschluss bieten, diese biete eine einfachen Handhabung, sind jedoch 3 mal so teuer, wie das Sportarmband.

An der Verbindung zwischen Armband und Uhr ist eine nicht weiter spezifizierter Wartungsport verbaut. Dieser Anschluss könnte in Zukunft Armbänder ermöglichen, welche Informationen aus dem Armband an die Uhr weiterleiten.


