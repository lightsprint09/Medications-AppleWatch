%%
% ----------------------------------------------------------------------------
% "THE BEER-WARE LICENSE" (Revision 42):
% <sebastian.rauh@hs-heilbronn.de/michael.bauer@hs-heilbronn.de> wrote this 
% file. As long as you retain this notice you can do whatever you want with 
% this stuff. If we meet some day, and you think this stuff is worth it, you 
% can buy us a beer in return. 
% Michael Lukas Bauer, Sebastian Felix Rauh
% ----------------------------------------------------------------------------
%%

\section{Evaluation}

Die Evaluierung des Prototypen soll eine quantifiziertes Ergebnis liefern, anhand dessen Schwachstellen aufgezeigt werden sollen. Diese Schwachstellen sollen mit der nächsten Interation der Software ausgebessert werden.

\subsection{Zielgruppe}
Die Zielgruppe sollten Menschen sein, deren Alltag von Medikamenteneinnahmen geprägt ist. Es sollte sich um Menschen mit wenig oder keinen Vorkenntnissen mit Touchscreen-basierten Geräten handeln.  Patienten fortgeschrittenen Altern, die sich stationär im Krankenhaus aufhalten eignen sich gut, Medikamente nehmen und viel freie Zeit während des Aufenthaltes im Krankenhaus für eine Befragung haben.

\subsection{Vorgehen}
Im ersten Schritt wird der Zielgruppe die Apple Watch gezeigt und die dahinterliegende Technologie erklärt, um Neugier bei den Patienten zu wecken. Nun können die Patienten die Uhr anlegen. 
Im zweiten Schritt wird ein Notification, die eine Vibration auslöst gestartet. Die Patienten müssen die Vibration spüren und dann auf die Uhr schauen. Dort sollen sie den Bildschirminhalt wiedergeben. Haben sie den Bildschirminhalt wiedergegeben, so müssen sie die Benachrichtigung bestätigen, also auf den Button mit der Aufschrift "Genommen" tippen.

In Schritt Drei müssen die Patienten durch drücken auf die digital Crown den App Bildschirm aufrufen um dort die Mediwatch App zu öffnen. Wenn die App geöffnet ist, soll ein Medikament ausgewählt werden und dieses um eine definierte Zeit verschoben werden. Ist das Medikament verschoben, so ist die Aktion beendet.

\subsection{Auswertung}
Für die Evaluation werden 2 Fragebögen genutzt. Einmal AttrakDiff\cite{UserInDe:Attrakdiff}, welcher die subjektiv  Wahrnehmung der Bedienbarkeit und Aussehen des Prototypen erfragt. Es handelt sich um einen standardisierten Fragebogen und eine standardisierten Auswertungsmethode. Da dieser Frageboden keine Antworten über Funktionen des Prototyp gibt, ist es nötig einen zweiten Fragebogen zu erstellen. Dieser Fragebogen erfragt die Situation, also den Kontext in dem sich der Anwender befindet und die daraus Folgenden Ansprüche an den Prototypen. So sollen fehlende Funktionen oder Fehler in der Analyse aufdecken werde. Die Fragen sind im Anhang zu finden.
Die Erfassung der Fragen wird mit Limesurvey\cite{Limesurvey} in Version 2.05 durchgeführt. Da sich Limesurvey selbst hosten lässt, bleiben die erfragten Daten auf einem sicheren Hochschulserver und werden nicht bei Drittanbietern gespeichert. 

\section{Durchführung der Evaluation}
\subsection{Beschreibung der Zielgruppe}
Es haben insgesamt 11 Patienten  an der Befragung teilgenommen. Davon waren 7 weiblich und 4 männlich. Alle waren im Alten zwischen 70 und 85. 

\subsection{Befragung der Patienten}
Die Befragung in der Zielgruppe verlief sehr schwierig und nicht wie geplant.
Der erste Schritt der Befragung klappte bei fast allen Probanden sehr gut. So erkannten alle die Notification mit der einhergehenden Vibration wurde von allen Patienten als verständlich geschildert. Das darauf folgende Bestätigen einer Notification verlief zu großen Teilen schlecht. Die Patienten konnten die Buttons zum Bestätigen nicht treffen, da ihnen die Feinmotorik fehlte.
Öffnen einer App ist auf Grund der fehlenden Feinmotorik ebenfalls nicht möglich. Die App-Icons sind zu klein und werden nicht getroffen. Das öffnen der Anwendung wurde bei allen Patienten nicht erreicht.
\subsection{Auswertung}
Der Fragebogen AttrakDiff ist in dieser Zielgruppe nicht praktikabel. Eine sehr genauer Unterscheidung zwischen den Begrifflichkeit, die der  AttrakDiff abfragt ist für die Zielgruppe nicht möglich. Dies fällt auf, wenn man mit den Probanden spricht. Die Probanden schweifen oft ab und es ist nicht möglich ihre Aussagen zu erfassen. Direkte Fragen, die den Alltag der Probanden betreffen werden zuverlässig beantwortet. Schwierigkeiten, die bei der Medikamenten Einnahme werden nicht zugegeben.
Die Befragung wurden deswegen auf eine Thinking Alound Methode \cite{Sommerville:2016aa} umgestellt. Leider sind die Ergebnisse so nicht mehr quantifizierter, jedoch sind aus des Aussagen der Patienten und den Beobachtungen gute Schlussfolgerungen möglich.
Auch der der zweite Fragebogen zu den Funktionen und deren Nutzungskontext wurde nicht genutzt. Hier wurde versucht 
Während er Gesprächs  mehr Einblick in das Leben der Probanden zu erhalten. Probanden erzählen gerne über ihr Leben und suchen eher das persönliche Gespräch. Aus diesen Gesprächen geht hervor, dass die Patienten mit großer Übereinstimmung keinen Internetanachluss besitzen. Dies schließt entfernte Aktualisierungen des Mediktionsplans und Einnahme Bestätigungen für den Arzt aus. Bedenken äußern die Weiblichen Patientinnen über die Größe und das Aussehen der Uhr. Eine Uhr muss dem Geschmack der Patientin gefallen. Hierbei spielt die Größe der Uhr, wie auch das Aussehen eine Rolle. Bei Beschreibung von anderen modischen Farben der Uhr oder der Armbänder, zeigen sie sich interessiert.  Ein Großteil der Patienten sieht jedoch ein, dass sie auf Grund Ihrer Sehschwäche die große Uhr (42mm) brauchen. Die große Uhr wird in Gegenzug als zu Groß beschrieben. Die meisten Patienten haben keine Probleme die Schrift zu lesen. Meist nehmen Sie Ihre Brille zur Hilfe, die im Krankenhaus natürlich immer griffbereit ist. Nur bei einem kleine Teil kommt es vor, dass sie den Text auf dem Display auch mit Brille nicht lesen können. Das größte Problem ist jedoch die schon genannte fehlende Feinmotorik. So können die Probanden die Uhr nicht wirklich aktiv bedienen, sondern reagieren nur auf die Vibration am Handgelenk. Dies führt zu einer Hilflosigkeit gegenüber der Technik der Uhr.
\subsection{Probleme tabellarisch}
\begin{table}[]
\centering
\caption{My caption}
\label{my-label}
\begin{tabular}{|l|l|l|}
\hline
 Problem  &Auswirkungen  &mögliche Lösung  \\ \hline
 fehlender Internetanschluss  &keine Aktualisierungen und keine Benachrichtigungen für den Arzt &Uhr oder Smartphone mit Mobilfunk  \\
 fehlende Feinmotorik der Probanden&  &  \\
 Vibration zu leicht&  &  \\
 Nur mit Brille lesbar&  &  \\
 Armband schwer anlegbar&  &  \\
 Uhr ist unästhethisch&  &  \\ \hline
\end{tabular}
\end{table}
