%%
% ----------------------------------------------------------------------------
% "THE BEER-WARE LICENSE" (Revision 42):
% <sebastian.rauh@hs-heilbronn.de/michael.bauer@hs-heilbronn.de> wrote this 
% file. As long as you retain this notice you can do whatever you want with 
% this stuff. If we meet some day, and you think this stuff is worth it, you 
% can buy us a beer in return. 
% Michael Lukas Bauer, Sebastian Felix Rauh
% ----------------------------------------------------------------------------
%%

\section{Evaluation}

Die Evaluierung des Prototyps soll ein quantifiziertes Ergebnis liefern, anhand dessen Schwachstellen aufgezeigt werden sollen. Diese Schwachstellen werden mit der nächsten Iteration der Software ausgebessert.

\subsection{Zielgruppe der Evaluation}
Die Zielgruppe sollten Menschen sein, deren Alltag von Medikamenteneinnahmen geprägt ist. Es sollte sich um Menschen mit wenig oder keinen Vorkenntnissen mit touchscreen-basierten Geräten handeln.  Patienten fortgeschrittenen Alters, die sich stationär im Krankenhaus aufhalten, eignen sich gut. Diese nehmen in der Regel Medikamente und haben viel freie Zeit während des Aufenthaltes im Krankenhaus für eine Befragung.

\subsection{Vorgehen}
Im ersten Schritt wird dem Probanden die Apple Watch gezeigt und die dahinterliegende Technologie erklärt, um Neugier bei dem Patienten zu wecken. Nun können die Patienten die Uhr selbstständig anlegen. 
Im zweiten Schritt wird eine Notification, die eine Vibration auslöst, gestartet. Die Patienten müssen die Vibration spüren und dann auf die Uhr schauen. Dort sollen sie den Bildschirminhalt wiedergeben. Haben sie diese Aufgabe erfolgreich gelöst, so müssen sie die Benachrichtigung bestätigen, also auf den Button mit der Aufschrift \glqq Genommen \grqq (siehe \ref{fig:watch-app-notification}) tippen.

In Schritt Drei müssen die Patienten den App-Bildschirm durch Drücken auf die Digital Crown aufrufen, um dort die Mediwatch App zu öffnen. Wenn die App geöffnet ist, soll ein Medikament ausgewählt werden und dessen Einnahmezeitpunkt um eine definierte Zeitdauer verschoben werden. Ist der Einnahmezeitpunkt verschoben, so ist die Aktion beendet.

\subsection{Auswertung}
Für die Evaluation werden zwei Fragebögen genutzt. Einmal AttrakDiff\cite{UserInDe:Attrakdiff}, welcher die subjektive  Wahrnehmung der Bedienbarkeit und das Aussehen des Prototyps erfragt. Es handelt sich um einen standardisierten Fragebogen und eine standardisierte Auswertungsmethode. Da dieser Fragebogen keine Antworten über Funktionen des Prototyps gibt, ist es nötig, einen zweiten Fragebogen zu erstellen. Dieser erfragt die Situation, also den Kontext in dem sich der Anwender befindet und die daraus folgenden Ansprüche an den Prototyp. So sollen fehlende Funktionen oder Fehler in der Analyse aufdeckt werden. Die Fragen sind im Anhang zu finden.
Die Erfassung der Fragen wird mit Limesurvey\cite{Limesurvey} in Version 2.05 durchgeführt. Da sich Limesurvey selbst hosten lässt, bleiben die erfragten Daten auf einem sicheren Hochschulserver und werden nicht bei Drittanbietern gespeichert. 

\section{Durchführung der Evaluation}
\subsection{Beschreibung der Zielgruppe}
Es haben insgesamt 11 Patienten  an der Befragung teilgenommen. Davon waren 7 weiblich und 4 männlich. Alle waren im Alter zwischen 70 und 85. 

\subsection{Befragung der Patienten}
Die Befragung in der Zielgruppe verlief sehr schwierig und nicht wie geplant.
Der erste Schritt der Befragung klappte bei fast allen Probanden sehr gut. So erkannten sie die Notification mit der einhergehenden Vibration. Der Bildschirminhalt wurde von fast allen Patienten als verständlich geschildert. Das darauf folgende Bestätigen einer Notification verlief zu großen Teilen schlecht. Die Patienten konnten die Buttons zum Bestätigen nicht treffen, da bei ihnen die Feinmotorik eingeschränkt war.
Das Öffnen einer App ist aufgrund der reduzierten Feinmotorik ebenfalls nicht möglich. Die App-Icons sind zu klein und werden nicht getroffen. Das Öffnen der Anwendung wurde bei allen Patienten nicht erreicht.
\subsection{Auswertung}
Der Fragebogen AttrakDiff ist in dieser Zielgruppe nicht praktikabel. Eine sehr genaue Unterscheidung zwischen den Begrifflichkeiten, die der  AttrakDiff abfragt, ist für die Zielgruppe nicht möglich. Dies fällt auf, wenn man mit den Probanden spricht. Sie schweifen oft ab und man ist nicht in der Lage, ihre Aussagen zu erfassen. Direkte Fragen, die den Alltag der Probanden betreffen, werden zuverlässig beantwortet. Schwierigkeiten, die die Patienten mit der Medikamenteneinnahme haben,  werden nicht zugegeben bzw. heruntergespielt.
Die Befragungen wurden deswegen auf eine Thinking Alound Methode \cite{Sommerville:2016aa} umgestellt. Leider sind die Ergebnisse so nicht mehr quantifizierbar, jedoch sind aus den Aussagen der Patienten und den Beobachtungen gute Schlussfolgerungen möglich.
Auch der zweite Fragebogen zu den Funktionen und deren Nutzungskontext wurde nicht verwendet. Hier wurde versucht, während des Gesprächs  mehr Einblick in das Leben der Probanden zu erhalten. Probanden erzählen gerne über ihr Leben und suchen eher das persönliche Gespräch. Daraus geht hervor, dass die Patienten mit großer Übereinstimmung keinen Internetanschluss besitzen. Dies schließt entfernte Aktualisierungen des Medikationsplans und Einnahmebestätigungen des Patienten für den Arzt aus. Missgefallen äußern die weiblichen Patientinnen über die Größe und das Aussehen der Uhr. Eine Uhr muss dem Geschmack der Patientin entsprechen. Hierbei spielt die Größe wie auch das Aussehen der Uhr eine Rolle. Bei Beschreibung von anderen modischen Farben der Uhr oder der Armbänder zeigen sie sich interessiert. Ein Großteil der Patienten sieht jedoch ein, dass sie aufgrund ihrer Sehschwäche die große Uhr (42mm) benötigen. Die große Uhr wird im Gegenzug als zu groß beschrieben. Die meisten Patienten haben keine Probleme die Schrift zu lesen. Meist nehmen sie ihre Brille zur Hilfe, die im Krankenhaus natürlich immer griffbereit ist. Nur bei einem kleinen Teil kommt es vor, dass sie den Text auf dem Display auch mit Brille nicht lesen können. Das größte Problem ist jedoch die schon erwähnte Einschränkung der Feinmotorik. So können die Probanden die Uhr nicht wirklich aktiv bedienen, sondern reagieren nur auf die Vibration am Handgelenk. Dies führt zu einer Hilflosigkeit gegenüber der Technik der Uhr.
\subsection{Probleme tabellarisch}
\begin{table}[]
\centering
\caption{Probleme und mögliche Lösungen des Smartwatch Prototyps}
\label{my-label}
\begin{tabular}{p{4cm} p{5cm} p{5cm}|l|l|l|}
\hline
 Problem  &Auswirkungen  &mögliche Lösung  \\ \hline
 \textbf{fehlender Internetanschluss}  &keine Aktualisierungen des Medikamentenplans und keine Benachrichtigungen für den Arzt &Uhr oder Smartphone mit Mobilfunkverbindung ausstatten  \\
 \textbf{reduzierte Feinmotorik der Probanden} &Buttons werden nicht getroffen oder falsche Aktionen werden ausgelöst  &große haptische Knöpfe in die Hardware integrieren, die gut drückbar sind  \\
 \textbf{Vibration zu leicht}&Vibration wird nicht gespürt und Medikament wird vergessen  &Stärkere Vibration, lauten Ton dazu abspielen, starkes optisches Feedback (grelles Blinken)   \\
 \textbf{Nur mit Brille lesbar} &Benachrichtigung wird erkannt, bis jedoch die Brille im Haushalt gefunden, ist die Medikation schon vergessen oder die Uhr hat die Benachrichtigung zurückgestellt  &Größere Schrift, die auch ohne Brille lesbar ist. Audiowiedergabe der Medikation  \\
 \textbf{Armband schwer anlegbar}&Uhr wird morgens nicht gerne angezogen,  bleibt liegen und Patient bekommt keine Benachrichtigungen  &Verzicht auf Standard Sport-Band und dafür Armbänder, die leicht anzuziehen sind, verwenden. Viele Patienten tragen dehnbare Bänder ohne Verschluss \\
 \textbf{Uhr ist unästhetisch}&Patient trägt die Uhr nicht und wird so nicht an die Medikamente erinnert  &Keine Standard Uhren kaufen, sondern den Patienten die Farbe und Form der Uhr und die Art des Armbandes aussuchen lassen  \\ \hline
\end{tabular}
\end{table}

\section{Zusammenfassung}
Ziel der Arbeit war es, die Akzeptanz eines Systems zur Erinnerung an Medikamenteneinnahme zu testen. Als Methode wurde eine Patientenbefragung verwendet. Patienten stehen dem Tragen einer Uhr am Handgelenk positiv gegenüber. Ob sie jetzt digital oder analog ist, ist nebensächlich. Aus diesen Gründen ist die Uhr ein ideales Gerät zur Erinnerung von Medikamenten. Die Apple Watch mit ihrer touchscreen-basierten Navigation stellt für die getestete Zielgruppe eine große Herausforderung dar. Reduzierte feinmotorische Fähigkeiten erschweren den Patienten die Interaktion mit der Uhr. Auch Vibration und Geräuschwiedergabe sind zu schwach und können häufig nicht wahrgenommen werden. Das ästhetische Aussehen der Uhr ist nicht zu vernachlässigen, da eine Uhr oft als eine Art Schmuck getragen wird. So trägt die Farbe, Form und Größe zur Akzeptanz der Uhr bei. Findet der Patient die Uhr unschön/unmodisch, so wird er sie nicht tragen und schlimmstenfalls seine Medikamente vergessen.

Da Smartwatches noch keine große Verbreitung haben, sind auch die Werkzeuge zur Umsetzung von Anwendungen begrenzt. Es gibt keine Werkzeuge zur Erstellung interaktiver Prototypen, die auf der Uhr ausführbar sind. Ein Prototyp kann ausschließlich unter Verwendung von Code erstellt werden. So vermehrt sich der Aufwand für schnelles iteratives Vorgehen. Das watchOS SDK von Apple bietet eine relativ übersichtliche Schnittstelle. Die Schnittstellen ermöglichen es komplexere Anwendungen zu erstellen. Diese sind jedoch in manchen Anwendungsgebieten noch limitiert. Da es sich um die erste Geräte-Generation handelt, muss abgewartet werden mit welchen Leistungsverbesserungen weitere Generationen nachgerüstet werden, um die Uhr zu einem wirklich nützlichen Werkzeug zu machen. Das Potential zu einem guten Nutzen zeigt die erste Generation auf, jedoch ist sie an manchen Stellen (Geschwindigkeit, lange Wartezeiten) noch nicht ausgereift.

Apples neue Programmiersprache Swift, die zur Implementierung genutzt wurde, bietet Konzepte moderner Programmiersprachen. Diese Konzepte unterstützen den Entwickler meist schon zur Compilezeit. Dies führt zu einer frühen Fehlererkennung. Durch diese Spracheigenschaften, wie Optionals (\ref{ch:optionals}), die Sicherheit in der Modellierung geben, oder Closures und First Class Functions (\ref{ch:closures}), die Konzepte funktionaler Programmierung bereitstellen, eignet sich Swift sehr gut als Lehrsprache. Dadurch, dass Swift unter einer Open Source Lizenz veröffentlicht wurde, kann man nun für mehrere Zielplattformen entwickeln.

\section{Ausblicke}
Mit den gewonnenen Ergebnissen kann dieses Projekt neu ausgerichtet werden. Zum Beispiel sollte die Zielgruppe über mehr Erfahrung mit digitalen Geräten verfügen. Der Einsatz der Uhr bei Kindern und Jugendlichen im Alter ab 12 Jahren ist denkbar. Diese könnten bei ihrer Therapie unterstützt und durch die moderne Technik, mit der sie aufgewachsen sind, motiviert werden Medikamente zeitgerecht einzunehmen.
Nachdem die Defizite mit der Bedienbarkeit durch die Evaluation entdeckt wurden, ergab die Recherche, dass die watchOS Plattform über Funktionen der Accessibility verfügt \cite{Apple:watchAccess} . Damit ist es möglich, Nutzungs-Erleichterungen für Menschen mit körperlichen Einschränkungen zu schaffen. So kann mit DynamicType, also eine dynamischen Schriftgröße in der App, die Lesbarkeit verbessert werden. Der Nutzer kann so seine eigene Schriftgröße wählen. Auch die Navigation zum Öffnen von Anwendungen wird mit Accessibility Einstellungen verbessert. Die Uhr als Hilfsmittel im Alltag zeigt gute Ansätze auf und ist in der Zukunft weiter zu verfolgen. Wichtig hierbei ist, dass die Uhr als solches bestehen bleibt und nicht durch einen großen unschönen Kasten am Handgelenk ersetzt wird.

