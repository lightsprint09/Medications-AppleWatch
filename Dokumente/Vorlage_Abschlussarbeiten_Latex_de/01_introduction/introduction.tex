\section{Motivation}
Durch steigenden Anzahl an Medikamenten, die Patienten über den Tag nehmen müssen, kann dies zu einer großen Mentalen Aufgabe für den Patienten werden. Durch Hilfsmittel wie Medikationsplänen oder nach Zeit vorsortierten Medikamente kann die Einnahme erleichtert werden. 

Zur Erleichterung der Patienten soll der Medikationsplan nun per Smartwatch einsehbar gemacht werden und somit dem Patient eine Interaktion mit dem Plan ermöglichen. Daraus bieten sich auch Vorteile für den behandelten Arzt, der Einblick in die Einnahmegewohnheiten seines Patienten bekommt. Initiale Ideen stammen auf dem PITA-Praktikum an der HS-Heilbronn. Hier wird auch in enger Zusammenarbeit mit dieser Arbeit ein Komponente für Ärzte entwickelt, die es ihnen ermöglicht die Medikationen der Patienten zu plegen, zu überwachen und auszuwerten.
\section{Zielsezung}
Die Ziele der Arbeit sind nachfolgend gelistet.

\begin{enumerate}
\item Entwickeln eines interaktives Prototypen, welcher auf der Apple Watch ausgeführt werden kann.
\item Evaluieren des Prototyps mit gegeigten Personen, die zur passenden Zielgruppe gehören.
\item Umsetzung der Ergebnisse der Evaluierung im Rahmen der technischen Möglichkeiten.
\item Anbinden des Prototypen an das Backend-System vom PITA-Praktikum
\end{enumerate}
\section{Aufbau der Arbeit}
\begin{itemize}
\item \textbf{Kapitel 2 - Grundlagen}
 
In Kapitel 2 nehmen wir Bezug auf das Forschungsumfeld der Arbeit. Weiter werden technologische Grundlagen beschrieben. Apple`s Swift Programmiersprache, sowie die Kommunikation der Apple Watch stehen dabei im Mittelpunkt. Desweiteren werden Grundlagen in der Evaluierung der Gebrauchstauglichkeit beschrieben
\item \textbf{Kapitel 3 - Analyse und Entwurf}

Hier werden Anforderung beschrieben, die teilweise aus vorherigem PITA-Praktikum stammen. Weiter werden Anforderungen geschildert, die sich aus den Möglichkeiten der Apple Watch ergeben.

\item \textbf{Kapitel 4 - Implementierung}

Im vierten Abschnitt betrachten wir Kernpunkte der Implementierung. Weiter wird hier der Prototyp beschrieben

\item \textbf{Kapitel 5 - Evaluierung}

In Kapitel 5 wird die Planung und Durchführung der Evaluierung beschrieben. Die  Ergebnisse der Evaluierung an der Zielgruppe sind hier zu finden.

\item \textbf{Kapitel 6 - Diskussion}

In der Diskussion wird aufgezeigt, welche Probleme während der Arbeit entstanden sind. Es wird gezeigt wie der Prototyp im Gesamtbild einzuordnen ist und es wird ein Ausblick gegeben, welche Ziele mit dem System weiter verfolgt werden können
\end{itemize}

